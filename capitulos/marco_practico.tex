\section{Entrenamiento de la red neuronal}
En este capítulo se desarrollaran los diferentes pasos para lograr entrenar una red neuronal, para que se capaz de detectar personas ya sea en imágenes y en videos tanto grabados como en tiempo real.
 \subsection{Base de datos}
Uno de los primeros pasos para entrenar una red neuronal es tener una base de datos, el cual esta compuesto por imágenes (donde aparezca el objeto a detectar) y un archivo que indique la posición del objeto a detectar en la imagen.
\\ Para ello se consiguió 400 imágenes donde aparezcan personas y con ellos realizar un procedimiento de etiquetado, en la figura \ref{fig:labeling} se observa este proceso. Esto se logra mediante el programa LabelImg que te permite etiquetar la posición de una persona en una imagen.


\begin{figure}[h]
	\centering
	\includegraphics[scale=0.25]{imagenes/persona1.png}
	\caption{Proceso de etiquetado para general los archivos necesarios para el entrenamiento.}
	\source{Elaboración propia}
	\label{fig:labeling}
\end{figure}

\clearpage

Este proceso se realiza con cada imagen, para así obtener un archivo que especifica la posición de dónde se encuentra la \say{persona} en este caso. En la figura \ref{fig:labeling2} se observa la manera en la que se guardan esos datos. 


\begin{figure}[h]
	\centering
	\includegraphics[scale=0.3]{imagenes/persona2.png}
	\caption{Coordenadas obtenidas por el programa LabelImg al etiquetar a un objeto.}
	\source{Elaboración propia}
	\label{fig:labeling2}
\end{figure}

Esta tarea dependiendo de la cantidad imágenes podrá ser muy tedioso 






