\section{Entrenamiento de la red neuronal}
En este capítulo se desarrollaran los diferentes pasos para lograr entrenar una red neuronal, para que se capaz de detectar personas ya sea en imágenes y en videos tanto grabados como en tiempo real.
 \subsection{Base de datos}
Uno de los primeros pasos para entrenar una red neuronal es tener una base de datos, el cual esta compuesto por imágenes (donde aparezca el objeto a detectar) y un archivo que indique la posición del objeto a detectar en la imagen.
\\ Para ello se consiguió 400 imágenes donde aparezcan personas y con ellos realizar un procedimiento de etiquetado, en la figura \ref{fig:labeling}} se observa este proceso. Esto se logra mediante el uso de un programa que te permite etiquet

\begin{figure}[h]
	\centering
	\includegraphics[scale=0.7]{imagenes/fila_larga.png}
	\caption{Representación de una fila larga.}
	\source{Shutterstock}
	\label{fig:labeling}
\end{figure}