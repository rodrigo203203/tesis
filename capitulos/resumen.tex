
\begin{center}
	\section*{RESUMEN}
\end{center}

Este proyecto consiste en construir una aplicación, el cual podrá calcular el tiempo promedio de espera de una persona cuando se encuentra haciendo fila en un sistema de fila única. Con esta información se podrá sugerir cuándo es necesario abrir una nueva caja para que los clientes no esperen demasiado y así generar una mejor experiencia.

\vspace{5mm}
Para realizar esta aplicación se está usando las tecnologías de visión computacional y redes neuronales artificiales, con las que se puede obtener la detección de objetos, que en este caso serán personas. 

\vspace{5mm}
En el caso de la red neuronal, se decidió usar YOLO (You Only Look Once), la cual según varios informes es una de las mejores al momento de hablar de detección de objetos en tiempo real. Para poder detectar personas, primeramente, se tuvo que entrenar esta red neuronal mediante una gran cantidad de imágenes donde aparecen personas.

\vspace{5mm}
El presente trabajo podrá ser utilizado en cualquier comercio que utilice un sistema de fila única, ya que fue diseñado para resolver un problema que surge, el cual consiste en cuantas cajas tienen que estar abiertas en los diferentes periodos en la atención.