\section{Estado del Arte}
\subsection{Evolución de los Sistemas de filas de un supermercado}
El primer supermercado se inauguró en el año 1916 en Tennessee, Estados Unidos, con el nombre Pegly Wiggly, pero presentaba un sistema de cajas que se concentraban más en facilitar y agilizar el trabajo del cajero, sin darle mucha importancia al cliente, ya que las personas no tenían mucho espacio de maniobra, como se puede observar en la Figura \ref{fig:qw}.

\vspace{5mm}
\begin{figure}[h]
	\centering
	\includegraphics[scale=0.3]{imagenes/pegly.jpg}
	\caption{Sistema de caja del primer supermercado.}
	\source{Xataka (2016).}
	\label{fig:qw}
\end{figure}

La gran innovación surgió cuando se incorporó el concepto de la teoría de colas, que fue introducido por Agner Kraup Erlang,  cuyo articulo estaba enfocado en las llamadas telefónicas. 
Teniendo los conceptos ya establecidos, se fueron probando diferentes sistemas para agilizar la atención, ya que cada vez mas gente realizaba sus compras en las tiendas. Estos modelos son: FIFO, LIFO y SJF (\cite{fifo}). 
\clearpage

La búsqueda de mejoras para agilizar la atención en las cajas es un hecho, ya que estudios demuestran la importancia de generar una gran experiencia de compra, porque eso aumenta la probabilidad que el cliente vuelva al local a realizar otra compra, (\cite{enjoy1}); (\cite{enjoy2}); (\cite{shoppingExp}); (\cite{enjoy3}).\\ 

Una de las nuevas innovaciones viene dada por parte de Amazon, donde abrió locales en el que no existen cajas, como se observa en la Figura \ref{fig:amago}. Este nuevo sistema que utiliza Amazon realiza un cobro mediante el uso de  cámaras con visión computacional para así detectar que productos selecciona el cliente, esos productos seleccionados se agregan directamente a un carrito virtual en la aplicación de Amazon, el cobro se realiza cuando el cliente sale del local (\cite{ama}); (\cite{ama2}).

\vspace{5mm}
\begin{figure}[h]
	\centering
	\includegraphics[scale=0.3]{imagenes/amago.jpg}
	\caption{Supermercado de Amazon que no utiliza cajas.}
	\source{Alicia Davara (2019).}
	\label{fig:amago}
\end{figure}

Otra innovación viene de los supermercados de China, donde implementaron un sistema de pago mediante el uso de la aplicación WeChat, en este caso los clientes tienen que escanear los diferentes productos que compraran y esto genera una \say{lista de compra}, la cual al momento de realizar el pago se crea un código QR, que posteriormente es escaneado por el cajero para registrar la compra y asei realizar el pago a travez de WeChat (\cite{wechat}), de esta manera se agiliza la atención y se evita que se genere largas filas. 
\clearpage
\subsection{Innovación en las redes neuronales}

 El concepto de las redes neuronales artificiales fue introducido por primera vez en 1943 por el matemático Walter Pitts y el neurólogo Warren McCulloch, mediante un articulo con el nombre \say{\textit{A Logical Calculus of Ideas Immanent in Nervous Activity}} (\cite{neu1}), en el cual se plantea un modelo computacional simplificado de como las neuronas biológicas podrían trabajar juntas, para poder realizar cálculos complejos utilizando modelos de lógica proposicional (\cite{neural}).
 
 El siguiente salto viene de la mano del psicólogo  Frank Rosenblatt, que por el año 1960 introdujo un nuevo concepto conocido como \say{perceptron} (\cite{neur2}), que permitía usar las redes neuronales como un clasificador binario (\cite{neu3}). Al principio solo podía representar funciones \textit{AND} y \textit{OR} pero en los siguientes años se pudo combinar varias redes neuronales que permitían representar más funciones lógicas.
 \vspace{5mm}
 
El dato sustancial para este proyecto es la introducción de las redes neuronales convolucionales o CNN (\textit{Convolutional Neural Network}) por sus siglas en inglés. El concepto viene por la investigación de Yann LeCun que permite trabajar con imágenes (\cite{neu4}); (\cite{neu5}). La manera en la que se usa estas redes se observa en la Figura \ref{fig:gato}. Este estudio genero muchas posibilidades de mejorar diversos sectores de la industria.
 
 \vspace{5mm}
 \begin{figure}[h]
	\centering
	\includegraphics[scale=0.2]{imagenes/gato.png}
	\caption{Aplicación de redes neuronales convolucionales para la clasificación de imágenes.}
	\source{Hamilton Chang (2019).}
	\label{fig:gato}
\end{figure}
En el proceso de la   se pasa por diferentes procedimientos, uno de ellos es la multiplicación con un \textit{kernel}, este proceso se apreciará mejor en la Figura \ref{fig:conv24}. En la Figura \ref{fig:conv23} observamos el resultado final de la convolución, donde a las entradas de la primera y última fila, y la primera y última columna se les ha asignado el valor original (\cite{conv2323}).
\clearpage
 \begin{figure}[h]
	\centering
	\includegraphics[scale=0.4]{imagenes/convu2.png}
	\caption{Procedimiento de convolución.}
	\source{Modelling in Science Education and Learning
Volume 9 (2016).}
	\label{fig:conv24}
\end{figure}
 \begin{figure}[h]
	\centering
	\includegraphics[scale=0.4]{imagenes/convu3.png}
	\caption{Resultado de una convolución.}
	\source{Modelling in Science Education and Learning
Volume 9(2) (2016).}
	\label{fig:conv23}
\end{figure}
Las redes neuronales convolucionales son la base de la mayoría de los proyectos que utilicen reconocimiento de objetos en imágenes (\cite{neu6}); (\cite{neu7}); (\cite{neu8}); Estos proyectos van desde el reconocimiento facial, clasificación de imágenes, detección de letras manuscritas y otros más. Las mejoras de esta red están enfocadas en aumentar la velocidad de respuesta y hacer más accesibles esta tecnología a más dispositivos, ya que un problema al trabajar con videos en tiempo real es la latencia. Este problema se solucionará con YOLO, gracias a su manera de aplicar CNN (\cite{yolo}), como se muestra en la Figura \ref{fig:yo}, se obtiene unos tiempos de ejecución favorables a este nuevo método. Donde COCO es una base de datos y está medido en porcentaje de acierto.
 \clearpage
 \begin{figure}[h]
	\centering
	\includegraphics[scale=0.2]{imagenes/image.png}
	\caption{Prueba de velocidad de YOLO con la base de datos COCO.}
	\source{YOLO (2018).}
		% TODO: Cambiar el caption
	\label{fig:yo}
\end{figure}
 
 
\section{Fundamentos Teóricos}
\subsection{Redes neuronales artificiales}

Las redes neuronales artificiales siguen el mismo funcionamiento que las neuronas biológicas, en el que las dendritas se encargan de captar los impulsos nerviosos que emiten otras neuronas, luego se las procesa y si es necesario se envía a otra neurona. En el caso de las neuronas artificiales, la suma de las entradas multiplicadas por el valor asociado determina el \say{impulso nervioso} que recibe la neurona. Este valor se procesa en el interior de la célula mediante una función de activación que devuelve un valor que se envía como salida de la neurona, esto se apreciará mejor en la Figura \ref{fig:natu} (\cite{open}). 

\vspace{5mm}
\begin{figure}[h]
	\centering
	\includegraphics[scale=0.216]{imagenes/natural}
	\caption{Red neuronal artificial en comparación a una neurona.}
	\source{Magiquo (2019).}
	\label{fig:natu}
\end{figure}
\vspace{4mm}

La palabra red, en el término \say{red neuronal artificial} se refiere a las interconexiones entre las neuronas en las diferentes capas de cada sistema. Una configuración ejemplar tiene tres capas. La primera posee neuronas de entrada que envían datos a través de las sinapsis a la segunda capa, y luego a través de más sinapsis a la tercera capa de salida. Los sistemas más complejos tendrán mayor número capas. Como se observa en la Figura \ref{fig:redneu} se presenta un ejemplo de cómo una neurona artificial tiene diferentes capas y cómo se conectan entre ellas.

\vspace{5mm}
\begin{figure}[h]
	\centering
	\includegraphics[scale=0.35]{imagenes/ela1}
	\caption{Estructura de una red neuronal artificial.}
	\source{Elaboración propia.}
	\label{fig:redneu}
\end{figure}

\subsubsection{Capas}

Para empezar una red neuronal es un conjunto de capas, y una capa se puede considerar como un conjunto de neuronas, en el cual sólo serán contadas a partir de la primera capa que sería la entrada.
Como se observa en la Figura \ref{fig:capa}, esa red neuronal tiene 3 capas, que son: la de entrada, la oculta y la de salida, pero en realidad sólo habría dos capas porque se descarta la capa de entrada. A simple vista, uno podría notar que las capas envían múltiples datos a la siguiente pero no es el caso, ya que dependiendo de la situación sólo una neurona enviará la información a la siguiente sección y de esa sección solo en una pasara la información, este procedimiento se repite hasta llegar a la salida.

\begin{figure}[h]
	\centering
	\includegraphics[scale=0.5]{imagenes/capa.png}
	\caption{Diferentes capas conectadas de una red neuronal.}
	\source{Elaboración propia}
	\label{fig:capa}
\end{figure}

\subsubsection{Métodos de aprendizaje}


Dependiendo del problema a resolver, el método de aprendizaje puede variar, de esta manera se podrá adecuar al proyecto de una mejor forma, estos métodos se distinguen en tres tipos de esquemas:

\subsubsection{Aprendizaje supervisado}


En el aprendizaje supervisado, los diferentes algoritmos trabajan con datos \say{etiquetados} (\textit{Labeled data}), que intentan encontrar una función, que dada a las variables de entrada \textit{(input data)}, se puede llegar asignar una etiqueta adecuada como se observa en la Figura \ref{fig:AprenSup}. El algoritmo se entrena mediante una base de datos y puede \say{aprender} a asignar las etiquetas de salida adecuadas (\cite{AprendizajeSup}).
\vspace{5mm}
\\
El aprendizaje supervisado se suele usar en:
\begin{itemize}
    \item Problemas de clasificación (identificación de dígitos, diagnósticos, o detección de fraude de identidad).
    \item Problemas de regresión (predicciones meteorológicas, de expectativa de vida, de crecimiento, etc).
\end{itemize}

\vspace{5mm}
\begin{figure}[h]
	\centering
	\includegraphics[scale=0.7]{imagenes/aprendizaje_supervisado.png}
	\caption{Aprendizaje supervisado.}
	\source{Diego Calvo (2016).}
	\label{fig:AprenSup}
\end{figure}
\clearpage

Estos dos tipos principales de aprendizaje supervisado, clasificación y regresión, se distinguen por el tipo de variable objetivo. En los casos de clasificación, es de tipo categórico, mientras que, en los casos de regresión, la variable objetivo es de tipo numérico.

Los algoritmos mas habituales al usar aprendizaje supervisado son:
 \begin{itemize}
    \item Árboles de decisión.
    \item Clasificación de Naïve Bayes.
    \item Regresión por mínimos cuadrados.
    \item Regresión Logística.
    \item Support Vector Machines (SVM).
    \item Métodos \say{Ensemble} (Conjuntos de clasificadores).
 \end{itemize}

\subsubsection{Aprendizaje reforzado}
El objetivo del aprendizaje reforzado, consiste que el mismo sistema pueda aprender y tomar decisiones por su misma experiencia. Es decir, dependiendo a la situación determinada el sistema va a seleccionar la opción optima, para así ejecutar un proceso. Esta respuesta mejorará con el tiempo médiate un procedimiento de prueba y error, que continuará hasta alcanzar al objetivo deseado (\cite{refor}).

\vspace{5mm}

Para realizar este entrenamiento se necesita de un “agente” en un estado determinado dentro de un ambiente, esta configuración se observa en la Figura \ref{fig:aprRefor}. Este método necesita una recompensa cada vez que alcance un objetivo, también se le dará una recompensa cuando alcance escenarios no deseados.

\begin{figure}[h]
	\centering
	\includegraphics[scale=0.5]{imagenes/apreRefor.png}
	\caption{Esquema de aprendizaje reforzado.}
	\source{Tao Bain (2017).}
	\label{fig:aprRefor}
\end{figure}

\subsection{Función de activación}
Ésta es otra similitud que tiene una red neuronal biológica con una artificial, esta función de activación es considerada como un impulso nervioso que es enviado al cerebro. Esta función de activación utiliza una suma ponderada de la entrada anterior.
\\

Existen varias funciones de activación que actúan de diferente manera y eso nos permite adecuar las capas de las neuronas, estas funciones son (\cite{funAc}):
\subsubsection{Sigmoid/Sigmoide}

Esta función se encarga de transformar los valores introducidos a una escala de una décima (0,1), la Ecuación (\ref{eq:sigmoideq}) tomará los valores de alto rango y lo transformara de manera asintomática a 1 y los valores bajos tienden de manera contraria acercándose al 0 (\cite{activa2}), la gráfica de la función se apreciará en la Figura \ref{fig:sigmoid}.

\vspace{5mm}
\begin{equation}
\label{eq:sigmoideq}
	f(x)=\frac{1}{(1-e^{-x})}
\end{equation}
\vspace{5mm}

Esta función se caracteriza en los siguientes factores :
\begin{itemize}
	\item Posee una lenta convergencia.
	\item Tiene un gran rendimiento en la ultima capa.
	\item Tiene buena acotación entre el 0 y 1.
\end{itemize}
\vspace{15mm}
\begin{figure}[h]
	\centering
	\includegraphics[scale=0.5]{imagenes/save.png}
	\caption{Representación gráfica de la función Sigmoid.}
	\source{Elaboración propia.}
	\label{fig:sigmoid}
\end{figure}

\clearpage
\subsubsection{Tangente hiperbólica(Tanh)}

Esta función tangencial consiste en transformar los valores de entrada en una escala (-1,1), donde los valores altos tenderán de manera asintomática a 1 y los valores de bajo nivel tenderán a 0, (\cite{activa2}). La gráfica de esta función se representará en Figura \ref{fig:tange}, la cual se obtiene con la Ecuación (\ref{eq:tangente}).
\begin{equation}
\label{eq:tangente}
	f\left( x\right) =\frac {2}{1+e^{-2x}}-1
\end{equation}


Las características de esta función son las siguientes:

\begin{itemize}
	\item Presenta una lenta convergencia.
	\item Se utiliza para decisiones de una sola opción.
	\item Tiene un gran rendimiento en redes recurrentes.
\end{itemize}

\vspace{5mm}
\begin{figure}[h]
	\centering
	\includegraphics[scale=0.15]{imagenes/tange.png}
	\caption{Representación gráfica de la función Tangencial.}
	\source{Travail personnel (2008).}
	\label{fig:tange}
\end{figure}

\subsubsection{Relu}

Esta función consiste en transformar todos los valores negativos a 0 y mantiene solo los positivos. Esta función de activación es la más utilizada a pesar de que existen otras más recientes (\cite{activa2}). La representación gráfica de esta función se encuentra en la Figura \ref{fig:relu} y se lo obtiene mediante la Ecuación (\ref{eq:relus}).

\vspace{5mm}
\begin{equation}
\label{eq:relus}
	F\left( x\right) =max\left( 0,x\right) =\begin{cases}0 para <0\\
x para\geq 0\end{cases}
\end{equation}
\vspace{5mm}

Algunas de las características de esta función son las siguientes:

\begin{itemize}
	\item Solo se activará cuando los valores sean positivos.
	\item Se comporta bien con imágenes.
	\item Tiene un gran desempeño con las redes neuronales.
\end{itemize}

\begin{figure}[h]
	\centering
	\includegraphics[width=9cm, height=5.9cm]{imagenes/relu.png}
	\caption{Representación gráfica de la función Relu.}
	\source{Visweswaran N. (2020).}
	\label{fig:relu}
\end{figure}



\subsubsection{Softmax}

La función Softmax consiste en calcular las probabilidades de un evento sobre \say{n} eventos diferentes. Es decir que esta función calculara las probabilidades de cada clase de objeto sobre todas las diferentes clases de objetivos posibles. El rango de esta función será de (0,1), y todas las sumas de las probabilidades serán igual a uno. Esta función se utiliza más que todo para obtener la probabilidad de modelos de multi clasificación (\cite{activa2}). La representación gráfica de la función está en la Figura \ref{fig:soft} y se la obtiene con la Ecuación (\ref{eq:softmax}).

\vspace{5mm}
\begin{equation}
\label{eq:softmax}
	f\left( z\right) _{j}=\frac {e^{zj}}{\sum ^{k}_{k=1}e^{z}k}
\end{equation}
\vspace{5mm}

La función de \textit{Softmax} se caracteriza por las siguientes razones:
\begin{itemize}
	\item Se utiliza cuando se quiere una respuesta en probabilidades.
	\item Se ocupa cuando se desea normalizar un tipo multiclase.
\end{itemize}

\begin{figure}[h]
	\centering
	\includegraphics[width=9cm, height=7cm]{imagenes/Softmax.png}
	\caption{Representación gráfica de la función Softmax.}
	\source{Saimadhu Polamuri (2016).}
	\label{fig:soft}
\end{figure}

\subsection{Inteligencia artificial}
La Inteligencia Artificial o IA es la combinación de diferentes algoritmos que con el tiempo se espera alcanzar o simular las capacidades del ser humano. La inteligencia artificial se puede considerar uno de los más grandes avances de la última década, ya que tiene diversas aplicaciones en diferentes áreas, por ejemplo, en la industria de la música se utiliza la IA para poder identificar el estilo musical que le puede gustar a una persona, por otra parte en la industrial automotriz ya existe autos que tiene la capacidad de que se manejen solos, un claro ejemplo son los vehículos de Tesla que cada vez van agregando mejoras a su sistema.

\vspace{5mm}
Cuando se habla de inteligencia artificial entramos al sector de ciencias de computación, donde esta área abarca los conceptos de \textit{machine learning, data mining y data science}, esta división se observa en la Figura \ref{fig:ia}.

\vspace{5mm}
 \begin{figure}[h]
	\centering
	\includegraphics[width=12cm, height=7cm, ]{imagenes/ia.jpg}
	\caption{División de los sectores en ciencias de computación.}
	\source{Machine learning and IA for Healthcare (2019).}
	\label{fig:ia}
\end{figure}
\vspace{5mm}

El concepto de la inteligencia artificial fue introducido en el año 1950 por Alan Turing, quien fue el primero en construir una máquina con la capacidad de pensar, la máquina que construyo se observa en la Figura \ref{fig:alan}, que fue conocido como \say{La máquina Colossus}, que tenia como objetivo poder descifrar los códigos enigma que utilizaban el ejército alemán en sus comunicaciones. También gracias a él, se creó una prueba conocida como \textit{The Turing Test}, el cual consiste en medir la capacidad que tiene una máquina en demostrar su \say{inteligencia artificial}, la prueba consiste en establecer una comunicación entre dos personas y una máquina, donde una persona no sabrá quien es la máquina o si hay una, si la persona no llega identificar a la máquina la prueba se considera como exitosa, ya que fue capaz de imitar el comportamiento humano (\cite{ia2}). 

\vspace{5mm}
 \begin{figure}[h]
	\centering
	\includegraphics[scale=0.3]{imagenes/alan1.jpg}
	\caption{La máquina Colossus creada por Alan Turing.}
	\source{BatallasHistoricas (2016).}
	\label{fig:alan}
\end{figure}

\subsubsection{Visión computacional}

La visión computacional es un campo que va muy conectado con la inteligencia artificial, que consiste que una máquina tenga un alto nivel de comprensión de imágenes o videos. Con esta tecnología se quiere automatizar tareas que requieran un sistema visual, por ejemplo, sistemas de clasificación de imágenes. Ahora, lograr una interpretación de imágenes al mismo nivel que el ser humano es un problema complejo. Sin embargo, en los últimos años hubo avances considerables, uno de ellos son el uso de esta tecnología para el piloto automático de los automóviles.
\subsubsection{OpenCV}
Actualmente, una de las herramientas más usadas para el campo de visión computacional es OpenCV, ya que es libre de uso y presenta varias opciones para trabajar en diversas aplicaciones que van desde reconocimiento de objetos, calibración de cámaras, visión estereoscópica y visión robótica.

\vspace{5mm}

OpenCV presenta una estructura modular, la cual dependiendo de la versión puede variar de paquetes que lo compongan, pero los principales son los siguientes:

\begin{itemize}
	\item Procesamiento de imagen.
	\item Análisis de videos.
	\item Calibración de cámara y reconstrucción 3D.
	\item Detección de objetos.
	\item Video I/O.
	\item Núcleo funcional.
\end{itemize} 

\subsection{Machine Learning}

El aprendizaje por máquina o \textit{machine learning} en inglés, consiste en poder extraer información de una base de datos y poder aprender de ella. El uso de \textit{machine learning} en los últimos años se hizo muy común, desde las páginas web como ser Facebook, YouTube o Amazon poseen algún tipo de \textit{machine learning}, ya que dependiendo del contenido que uno mire o producto que se busque generará información para qué la página aprenda y sepa que tipo de contenido recomendar (\cite{machine}).


El problema más común para el cual se utiliza \textit{machine learning}, es para la toma de desiciones, tomando en cuenta ejemplos de casos anteriores. Con esa información la \say{máquina} puede predecir cúal sería la salida dependiendo la entrada que se le entregue, esta respuesta que entrega podría ser una nueva creada por sí misma para satisfacer a la \say{pregunta}, todo esto lo realiza sin la intervención de un individuo.
\subsubsection{Tensorflow}


Tensorflow, es una biblioteca de software de código abierto para la computación numérica, esta herramienta fue desarrollado por Google. Tensorflow se creó para poder construir y entrenar redes neuronales, las cuales permite detectar, descifrar patrones y correlaciones en los datos que se están estudiando. En los últimos años, el interés por esta biblioteca que creó Google fue creciendo tal como se muestra en la Figura \ref{fig:datosT}, ya que es una herramienta muy útil para proyectos que tengan el uso de \textit{Machine learning}.

\begin{figure}[h]
	\centering
	\includegraphics[scale=0.35]{imagenes/tensor2323.png}
	\caption{Interés de búsqueda sobre Tensorflow en Google.}
	\source{Elaboración propia.}
	\label{fig:datosT}
\end{figure}

La plataforma tuvo sus inicios en el 2011 por parte del equipo Google Brain, que su lema se aprecia en la Figura \ref{fig:google22}, la primera versión de la plataforma era conocido con el nombre de DistBelief. Por el año 2014 el proyecto fue creciendo y tomando forma, así que se le cambió el nombre y ahí fue que le conoció como Tensorflow. Google liberó este software en su segunda versión el 9 de noviembre del 2015. Al pasar el tiempo esta herramienta se convirtió el más usado en el campo de Deep Learning, y eso se debe por la filosofía de Google: «código primero, código siempre».

\vspace{5mm}
\begin{figure}[h]
	\centering
	\includegraphics[scale=0.4]{imagenes/googleTeam.png}
	\caption{Equipo que desarrollo Tensorflow.}
	\source{Google.}
	\label{fig:google22}
\end{figure}

Para el año 2016, Google anunció una nueva función con el nombre de TPU o unidad de procesamiento de tensor, el cual es una construcción de un circuito integrado de aplicación específica o ASIC por sus siglas en inglés. Este circuito permite realizar un aprendizaje automática y que se adapta para Tensorflow. Esta implementación del TPU es un acelerador programable, con el fin de conseguir un alto \textit{throughput} (número de mensajes recibidos con éxito), y para correr modelos más rápidos al generar un entrenamiento. 

\subsubsection{Gráficos de cómputo}

Usando Tensorflow nos permite crear algoritmos de aprendizaje que puedan interactuar entre sí, y estas interacciones se los llega a conocer como grafo computacional.
Al referirse a un grafo en este tema, se quiere decir a un conjunto de nodos conectados entre sí, estos nodos normalmente tienen entrada y salida al mismo tiempo, cada nodo tiene a dentro alguna función aritmética, desde las más simples como ser sumas y multiplicaciones hasta las más complejas (\cite{tensor1}).

Para la creación de un grafo se usará un ejemplo básico del libro \textit{Learning Tensorflow} (\cite{tensor1}), Para empezar, se importará la librería al editor mediante el primer comando que se aprecia en el Código \ref{55}, siguiendo con esto se crearan unos 3 nodos con diferente valor.

\vspace{5mm}
 \begin{lstlisting}[language=Python,caption={Ejemplos de los primeros pasos para crear nodos con Tensorflow.},captionpos=b,label=55]
	import tensorflow as tf
	a = tf.constant(5)
	b = tf.constant(2)
	c = tf.constant(3)
\end{lstlisting}

Para la siguiente etapa se creará otros 3 nodos pero esta vez tendrán ecuaciones que se ejecuten dentro ellas y además uno de esos nodos tendrá una variable que no se definió anteriormente, la escritura del código se podrá apreciar en el Código \ref{56}.

 \begin{lstlisting}[language=Python,caption={Ejecución de funciones aritméticas en Tensorflow.},captionpos=b,label=56]
	d = tf.multiply(a,b)
	e = tf.add(c,b)
	f = tf.subtract(d,e)
\end{lstlisting}


La representación gráfica de las funciones creadas mediante código se podrán ver en la Figura \ref{fig:capa3}, cómo se puede observar los nodos se conectan entre sí y por ellos mismos crean otro nodo resultante de la operación del nodo \say{e} con el \say{d}, para así crear el nodo \say{f}. 

\begin{figure}[h]
	\centering
	\includegraphics[scale=1]{imagenes/capa3.png}
	\caption{Ejecución de funciones aritméticas en Tensorflow.}
	\source{Learning Tensorflow (2017).}
	\label{fig:capa3}
\end{figure}

\subsubsection{Keras}

Es una extensión de alto nivel de TensorFlow que permite construir y entrenar modelos de aprendizaje profundo, esta modificación hace que TensorFlow sea más fácil de usar sin sacrificar la flexibilidad y el rendimiento. Keras contiene varias implementaciones de los bloques constructivos de las redes neuronales como por ejemplo los \textit{layers}, función de objetivo y función de activación.

Algunas ventajas al usar esta herramienta son las siguientes:
\begin{itemize}
    \item Keras presenta una interfaz simple y optimizada para diferentes casos de uso para el usuario, ya que proporciona información clara sobre errores que se podrían presentar al correr algún programa.
    \item Los modelos que se pueden crear con Keras son mediante bloques de conexión que se pueden configurar con pocas restricciones.
    \item Presenta una gran facilidad al momento de crear bloques, capas, métricas y diferentes funciones.
 \end{itemize}

\subsubsection{Configuración de capas iniciales}

Uno de los elementos principales al momento de crear una red neuronal son las capas que lo componen, éstas tienen como función de extraer la información del conjunto de datos que se están utilizando. Hay que tomar en cuenta que esa información extraída no siempre será útil para el problema que se quiere solucionar.


Al crear estas capas con Keras se necesita iniciar con el siguiente comando “tf.keras.layers.Flatten” si se quiere trabajar con análisis de imágenes, que consiste en transformar una imagen en un formato adecuado para que se pueda trabajar, con esto quiero decir que por ejemplo teniendo una imagen de un tamaño de 75x75 pixeles, al ejecutar el comando esta imagen se transformaría en 5625 pixeles (75 * 75 = 5625). Esta acción de “aplanar” los píxeles es el trabajo de la primera capa que se utiliza, con esta capa no comenzara con él aprendizaje sólo adecuará los sets de datos.

\vspace{5mm}

Para la siguiente etapa se tiene que aplicar una nueva capa con 2 diferentes configuraciones, para la primera capa se definirá con 128 nodos o neuronas y con la función de activación “relu“, par la siguiente capa solo tendrá 10 nodos y una función de activación “softmax”, estos comandos se observarán en el Código \ref{57}.

 \begin{lstlisting}[language=Python,caption={Configuración con Keras para trabajar con imágenes.},captionpos=b,label=57]
model = keras.Sequential([
	keras.layers.Flatten(input_shape=(75,75)),
	keras.layers.Dense(128, activation='relu'),
	keras.layers.Dense(10, activation='softmax')
])
\end{lstlisting}

\subsection{Deep Learning}

Como su nombre indica \textit{Deep learning} es la parte profunda del aprendizaje de una inteligencia artificial, esto se entenderá mejor en la observando la Figura \ref{fig:deep1}. Entre mayor sea la cantidad de capas relacionado a un modelo de se lo conoce como \textit{\say{deep}} (\cite{deep23}).
\clearpage
 \begin{figure}[h]
	\centering
	\includegraphics[scale=1]{imagenes/deep2.jpg}
	\caption{Posición de Deep Learning en una inteligencia artificial.}
	\source{Machine learning and IA for Healthcare (2019).}
	\label{fig:deep1}
\end{figure}

Los datos de entrada que almacena una capa se lo conocen como \say{weights}, que en esencia son un montón de números. En términos más técnicos \say{weights} se lo conoce como parámetros de una capa.

\vspace{5mm}

 Así que este aprendizaje consiste en encontrar valores que satisfagan a los parámetros establecido en el \say{weights} de la red neuronal. 
 Pero estos parámetros pueden ser mas de  un millón, así que encontrar un valor que satisfaga a cada parámetro puede parecer tedioso, más aún que cuando se encuentra un valor eso afecta a los demás.


\subsection{YOLO}

YOLO (\textit{You Only Look Once}, \say{sólo se ve una vez}) es un algoritmo de visión artificial que detecta y clasifica objetos en tiempo real, que permite detectar diferentes objetos que se pueden presentar en una imagen y poder enmarcarlos con un cuadro alrededor de los objetos encontrados Figura \ref{fig:yolo1}.

\begin{figure}[h]
	\centering
	\includegraphics[scale=0.8]{imagenes/yolo1.png}
	\caption{Detección múltiple realizada por YOLO.}
	\source{YOLO (2018).}
	\label{fig:yolo1}
\end{figure}

La principal innovación que YOLO trajo consigo a este campo, fue el hecho de que es capaz de realizar múltiples detecciones de una sola vez, lo cual realiza este proceso de manera rápida y eficiente.
El procesamiento de imágenes con YOLO es simple y directo, por ejemplo, teniendo una imagen de entrada de un tamaño de 448x448ppx, esto se aprecia en la Figura \ref{fig:yolo2}, ejecuta una red convolucional a la imagen que llegaría a cambiar de tamaño y así limitar los resultantes de las detecciones confiando en el modelo que se usó para entrenar la red neuronal. 
\begin{figure}[h]
	\centering
	\includegraphics[scale=0.8]{imagenes/yolo2.png}
	\caption{Aplicación de YOLO en una imagen.}
	\source{YOLO (2018).}
	\label{fig:yolo2}
\end{figure}

Entrando más a detalle, YOLO primeramente empieza a cuadricular la imagen que se está usando para la detección, seguido de eso va dibujando unas \say{cajas delimitadoras} mediante la guía del modelo entrenado, este modelo puede tener diversas etiquetas para la identificación de objetos y determinará cuando lleguen a un porcentaje de confianza definido, todos esos pasos se pueden apreciar en la Figura (\ref{fig:yolo4}). 

 \begin{figure}[h]
	\centering
	\includegraphics[scale=0.6]{imagenes/yolo3.png}
	\caption{Proceso que realiza YOLO para generar una detección de uno o varios objetos dentro de una imagen.}
	\source{YOLO (2018).}
	\label{fig:yolo4}
\end{figure}

La estructura de la red neuronal YOLO esta basada en GoogLeNet, la cual es otra red neuronal que utiliza la estructura \textit{Inception}. Este tipo de estructura permite hacer uso de multiples filtros de diversos tamaños en una sola capa, con esto permite que la red se adapte mejor a los diversos casos de detección \cite{googlenet}.

Internamente YOLO está compuesto por 24 capas convolucionales y por 2 capas que se encuentran completamente conectadas, La estructura completa se puede observar en la Figura \ref{fig:yolo108}.


 \begin{figure}[h]
	\centering
	\includegraphics[scale=0.3]{imagenes/yolo-interno.png}
	\caption{Estructura interna de la red neuronal YOLO.}
	\source{YOLO (2016).}
	\label{fig:yolo108}
\end{figure}

Esta estructura por tener una gran cantidad de capas y filtros posee una gran detección, pero a un costo bastante en el rendimiento, sobre todo cuando se utiliza en computadoras que no poseen algún procesamiento gráfico (GPU), debido a esto surgen otras versiones de YOLO.


\subsubsection{Tiny Yolo}

Tiny Yolo es una versión reducida de YOLO, ya que su estructura esta compuesta por 7 capas convolucionales y 6 capas con la función \textit{max-pooling}, estas últimas capas tiene la función de resaltar los valores máximos encontrados en la imagen (\cite{tiny2}).

Al tener menos capas está red neuronal es capaz de tener un mejor rendimiento en computadoras que no cuenten con una GPU, pero por eso mismo la red pierde precisión al momento de detectar a objetos. La estructura de esta red neuronal se observa en la Figura \ref{fig:yolo109}.

 \begin{figure}[h]
	\centering
	\includegraphics[width=13cm, height=5.2cm]{imagenes/tiny-yolo.jpg}
	\caption{Estructura interna de la red neuronal Tiny YOLO.}
	\source{Wei Fang (2019).}
	\label{fig:yolo109}
\end{figure}

\subsubsection{Darknet}

Darknet es un \textit{framework} de \textit{open source} (código abierto) que permite trabajar con la red neuronal YOLO, este \textit{framework} esta escrita en C/CUDA. La finalidad de utilizar Darknet es poder entrenar la red neuronal para que tenga la capacidad de detectar personas y con ello poder utilizarlo en el proyecto.
\subsubsection{Darkflow}

Darkflow es una herramienta muy similar a Darknet, pero con la diferencia de que Darkflow utiliza Tensorflow como motor para la red neuronal, lo cual provoca que para usar esta red se tiene que realizar en un entorno que reconozca el lenguaje Python.

 


\subsection{Servicios de cómputo en la nube}

Computación en la nube es un término que normalmente se utiliza para denominar cualquier servicio que realice algún tipo de trabajo a través de internet. Para contratar un servicio de este estilo se tiene que hacer por AWS (\textit{Amazon Web Service}), Microsft Azure o Google Cloud.



\subsubsection{Amazon Web Service}

\textit{Amazon Web Services} o AWS en sus siglas en inglés, es un servicio en la nube que ofrece más de 175 servicios de manejo de datos a nivel global, estos servicios se pueden observar en la Figura \ref{fig:aws}. Ofrece tecnologías desde infraestructura para almacenamiento, servicios de cómputo hasta nuevas tecnologías como ser el aprendizaje automático e inteligencia artificial. 

 \begin{figure}[h]
	\centering
	\includegraphics[width=12cm, height=7cm, ]{imagenes/ama23.png}
	\caption{Servicio que ofrece AWS.}
	\source{Amazon (2019).}
	\label{fig:aws}
\end{figure}


Ofrece tres tipos de servicios principales en la nube, el cual son: infraestructura como servicio, plataforma como servicio y software como servicio. Cada tipo de servicio tiene diferentes niveles de control.

\vspace{5mm}

Se utilizó este servicio para crear una máquina en la nube que tenga las capacidades para correr el entrenamiento y que éste pueda estar encendido las 24 horas al día o por lo menos hasta que finalice el entrenamiento.

\vspace{5mm}

Para ser más específico el tipo de servicio que se utilizó es \textit{Amazon Elastic Compute Cloud} o EC2 por sus siglas en inglés, el cual es un servicio web que ofrece la capacidad de realizar cálculos de cómputo adaptable, que en otras palabras se refiere a hospedar sistemas de \textit{sotfware} para la ejecución de aplicaciones. 


\section{Google Colab}

Google Colab es una plataforma de programación a través de la nube, la ventaja de este servicio es que ofrece el uso de GPU para el entrenamiento de redes neuronales de manera gratuita, pero sólo se le puede dejar 5 horas en \textit{stand by} antes de que se cierre. Para usar este servicio lo único que se necesita es una cuenta en Google e iniciar sesión.

\section{Conclusiones del Capitulo II}

En el presente capítulo se presentaron los fundamentos teóricos relacionados con redes neuronales convolucionales y la inteligencia artificial. Los temas que se abordaron son las diferentes maneras en la que se puede crear una red neuronal y estructura que maneja YOLO. También se explica el uso de herramientas en la nube, la cual pueden facilitar el proceso de entrenamiento de una red neuronal.



















