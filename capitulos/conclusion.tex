\begin{itemize}
	\item Se realizo la investigación de diferentes métodos para construir una red neuronal para la detección de objetos en imágenes, como Tensorflow, Pytorch y Keras y se concluyo que lo mejor es usar una red ya construida que esta basada en Tensorflow, el cual es Darknet este nos da resultados muy superiores que al construir una.
	\item Se logró entrenar una red neuronal para que sea capaz de detectar a personas tanto en imágenes como en videos en tiempo real.
	\item Se junto la detección con la función de rastreo, el cual elimina ciertas fallas que tiene al trabajar con sólo la detección de objetos. Para la función de rastreo se decidió utilizar Deepsort, el cual es uno de los más conocidos ya que tiene uno de los mejores rendimientos al momento de realizar un rastreo de los objetos detectados. 
	\item Mediante el añadido del rastreo se puede obtener ID independientes de cada detección el cual permite poder contar los objetos que están siendo detectados y con ello poder sacar un tiempo promedio de espera.
	\item Como conclusión, se llega a obtener el resultado esperado al identificar a personas y con ello poder calcular el tiempo promedio de espera, para poder notificar al encargado cuando una persona esté esperando más que el tiempo máximo definido, para así abrir una nueva caja para mejorar la velocidad de atención.
\end{itemize}