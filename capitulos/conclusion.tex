\section{Conclusiones} 

\begin{itemize} 

	\item Se realizo la investigación de diferentes algoritmos para la detección de objetos en imágenes y videos, y se seleccionó el algoritmo de YOLOv4 porque su velocidad de detección es aceptable para los requerimientos de proyecto y no compromete demasiado la precisión al identificar a las personas. 

	\item Se entreno una red neuronal capaz de detectar personas en tiempo real utilizando imágenes y videos capturados por las cámaras de seguridad del supermercado. La red neuronal entrenada demostró tener una precisión del 71\% y un rendimiento del 63\% lo que es bastante aceptable tomando en cuenta que solo se utilizó 500 imágenes para el entrenamiento que se obtuvieron de las cámaras de seguridad. Esto se puede mejorar ampliando la cantidad y variedad de imágenes para el entrenamiento además que se identificó que el ángulo en la que se encuentra la cámara dificulta la detección de personas. 

	\item Se logro calcular el tiempo promedio de espera de las personas generando identificadores independientes para cada detección, esto permite poder contar los objetos que están siendo detectados y con ello poder calcular el tiempo promedio de espera. Para eliminar ciertas fallas que se tiene al trabajar solamente con la detección de objetos se agregó la función de rastreo Deepsort, el cual es uno de los más conocidos, ya que tiene uno de los mejores rendimientos al momento de realizar un rastreo de los objetos detectados. 

	\item Se implemento la aplicación de escritorio Neovision utilizando Python y Tkinter que obtiene las imágenes a través de OpenCv  para después ser analizadas por la red neuronal de YOLO que detecta a las personas y luego generar con un identificador único para cada persona. Con esta información la aplicación calcula el tiempo de espera promedio y presenta una alerta cuando se sobrepasa el tiempo máximo configurado.  

\end{itemize} 

\clearpage 

 

\section{Recomendaciones}  

Como recomendaciones técnicas, un aspecto muy importante es contar con el hardware adecuado para este tipo de sistemas, es decir, contar con una tarjeta de video (GPU) en la computadora donde se va a ejecutar el programa. El equipo que fue utilizado para este proyecto no cuenta con una GPU, con lo que conllevo a no tener el rendimiento adecuado, ya que todos los cálculos fueron ejecutados en el CPU. 

Otro apartado importante es tener en cuenta el tipo de cámara que se usará, ya que dependiendo de la calidad del video la exigencia de la aplicación será mayor, es decir, que si se trabaja con videos en 720p (1280x720 pixeles) consumirá menos recursos de la computadora que al trabajar con un video en 1080p (1920x1080 pixeles), porque la imagen es más pequeña y eso genera menos información al procesarlo.

Un apartado muy importante es definir bien la posición de la cámara de donde se sacará el video, ya que el lugar donde se coloque la cámara podría ayudar o perjudicar bastante al momento que la red neuronal realice las detecciones. 

Por otra parte, al momento de realizar el entrenamiento se recomienda usar una base de datos muy amplia y variada que le permita así obtener mejores resultados.

\section{Trabajos futuros} 

Uno de los trabajos a futuros es hacer uso de una base de datos que se almacene en la nube y no de manera local, esto facilitaría poder conocer los datos registrados, ya que se podría acceder desde cualquier computadora y así uno no se limita a usar la misma computadora para poder analizar los datos.   

Otro trabajo a futuro es la implementación de una aplicación móvil que se conecte con la base de datos que se encuentre en la nube, esta app notificaría a los trabajadores del supermercado cada vez que registre una alarma y con ello se podría solucionar más rápido el problema. 





