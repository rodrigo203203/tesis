\documentclass[12pt,twoside]{report}
\usepackage[utf8]{inputenc}
\usepackage{graphicx}
\graphicspath{imagenes/}
\usepackage[spanish]{babel}
\usepackage{float, graphicx, caption, color, amsmath}

\usepackage[hidelinks]{hyperref}
\usepackage{afterpage}
%\usepackage[a4paper,width=150mm,top=25mm,bottom=25mm,bindingoffset=6mm]{geometry}
\usepackage[a4paper, left=20mm, right=30mm, top=30mm,bottom=20mm]{geometry}
\usepackage{dirtytalk}
\usepackage{mathptmx}
\usepackage{multicol}
\usepackage{url}
\usepackage{fancyhdr}
\usepackage{caption}
\captionsetup[figure]{labelfont={bf}}
\newcommand{\source}[1]{\caption*{\textbf{Fuente:} {#1}} }
\usepackage{setspace}
\usepackage{enumitem}
\onehalfspacing

%Import the natbib package and sets a bibliography  and citation styles
\usepackage{natbib}
\bibliographystyle{abbrvnat}
\setcitestyle{authoryear,open={(},close={)}}

\fancyhf{}
\renewcommand{\headrulewidth}{0pt}
\fancyfoot[LE,RO]{\thepage}
\pagestyle{fancy}

\title{
    	\includegraphics[scale=0.4]{imagenes/LogoCatolica} \\
    	\vspace{10mm} 
    	\large{\textbf{Rodrigo Morales Rivas}} \\
    	\vspace{5mm}
    	\textbf{RECONOCIMIENTO DE PERSONAS MEDIANTE REDES NEURONALES CONVOLUCIONALES YOLO PARA LA OPTIMIZACIÓN DE FILAS DE LOS SUPERMERCADOS} \\
    	\vspace{10mm}
    	\begin{multicols}{2}
    	\vfill\null
    	\columnbreak
    	\small{Proyecto de grado presentado a la Carrera de Ingeniería Mecatrónica para obtener su habilitación a Taller de Grado I} \\
    	\end{multicols}
    	\vspace{19mm}    	
    	\small{Santa Cruz - Bolivia \\ Julio - 2020}
}
\date{\vspace{-5ex}}

\begin{document}


\maketitle
\afterpage{\thispagestyle{empty}\null\newpage}
\clearpage
\begin{center}
	\textbf{Rodrigo Morales Rivas} \\
    \vspace{5mm}	
	\textbf{RECONOCIMIENTO DE PERSONAS MEDIANTE REDES NEURONALES CONVOLUCIONALES YOLO PARA LA OPTIMIZACIÓN DE FILAS DE LOS SUPERMERCADOS}
	
	\vspace{15mm}
	
	Trabajo Dirigido fue sometido a análisis y defendido ante el tribunal compuesto por: \\
	% TODO: Preguntar si tienen que estar las tres opciones...
	
	\vspace{30mm}

	\begin{tabular}{@{}p{.in}p{2in}@{}}
	& \hrulefill \\
	& Job Ledezma, PhD. \\
	& \textbf{DOCENTE DE TALLER DE GRADO I} \\
	\end{tabular}

	\vspace{30mm}

	\begin{tabular}{@{}p{.in}p{2in}@{}}
	& \hrulefill \\
	& Francisco Suares, Ing. \\
	& \textbf{TUTOR} \\
	\end{tabular}
	
	\vspace{30mm}

	\begin{tabular}{@{}p{.in}p{2in}@{}}
	& \hrulefill \\

	% TODO: Preguntar el nombre del docente
	& Erik Pozo, Msc. \\
	& \textbf{DOCENTE INVITADO} \\
	\end{tabular}

	\vspace{30mm}
	Santa Cruz de la Sierra, 7 de julio del 2020
	\thispagestyle{empty}
	\end{center}
	
\clearpage

\afterpage{\blankpage}
\begin{center}
	\section*{\normalsize{AGRADECIMIENTOS}}
\end{center}
\vspace{5mm}

Para mi familia que me brindaron todo su apoyo en este transcurso de mi carrera universitaria, a mis amigos y compañeros que me ayudaron a vencer varias dificultades brindando su apoyo y confianza.

Agradezco mucho por la ayuda de mis docentes de la universidad que me guiaron a este mundo de la ingeniería.


\clearpage


\begin{center}
	\section*{RESUMEN}
\end{center}

Este proyecto consiste en construir una aplicación, el cual podrá calcular el tiempo promedio de espera de una persona cuando se encuentra haciendo fila en un sistema de fila única. Con esta información se podrá sugerir cuándo es necesario abrir una nueva caja para que los clientes no esperen demasiado y así generar una mejor experiencia.

\vspace{5mm}
Para realizar esta aplicación se está usando las tecnologías de visión computacional y redes neuronales artificiales, con las que se puede obtener la detección de objetos, que en este caso serán personas. 

\vspace{5mm}
En el caso de la red neuronal, se decidió usar YOLO (You Only Look Once), la cual según varios informes es una de las mejores al momento de hablar de detección de objetos en tiempo real. Para poder detectar personas, primeramente, se tuvo que entrenar esta red neuronal mediante una gran cantidad de imágenes donde aparecen personas.

\vspace{5mm}
El presente trabajo podrá ser utilizado en cualquier comercio que utilice un sistema de fila única, ya que fue diseñado para resolver un problema que surge, el cual consiste en cuantas cajas tienen que estar abiertas en los diferentes periodos en la atención.

\clearpage

\begin{center}
	\section*{ABSTRACT}
\end{center}


This project consists of building an application, which will be able to calculate the average waiting time of a person when they are queuing in a single-file system. With this information, it will be possible to suggest when it is necessary to open a new box so that customers do not wait too long and thus generating a better experience.

\vspace{5mm}
This application uses technologies such as computer vision and artificial neural networks, which can be used in object detection, which in this case will be people. 

\vspace{5mm}
In the case of the neural network, it was decided to use YOLO (You Only Look Once), which according to several reports is one of the best when talking about object detection in real time, to being able to detect people first had to train this neural network through a large number of images where people appear.

\vspace{5mm}
This work can be used in any store that uses a single-queue system, since it was designed to solve a problem that arises, which is how many couters should be open in different periods during service hours.



\clearpage
\begin{center}
	\section*{Abreviaciones y acrónimos}
\end{center}
\vspace{15mm}
\textbf{Bokeh} =  Desenfoque.\\
\textbf{YOLO} = You Only Look Once.\\
\textbf{FIFO} = Primero en entrar, primero en salir (first in, first out).\\
\textbf{LIFO} = Ultimo en entrar, primero en salir (last in, first out).\\
\textbf{LSF} = Primero el trabajo más corto (shortest job first).\\
\textbf{CNN} = Convolution Neural Network.\\
\textbf{Sinapsis} = Región de comunicación entre la neurita o prolongación citoplasmática de una neurona.\\
\textbf{Espureas curiosas} = Correlaciones curiosas.\\
\textbf{TPU} = Unidad de procesamiento de Tensor.
\listoffigures
\thispagestyle{empty}
\tableofcontents
\thispagestyle{empty}

\chapter{INTRODUCCIÓN}
\fancypagestyle{plain}{
\fancyfoot[R]{\thepage}}
Las redes neuronales artificiales, hoy en día, se han convertido en un campo de estudio muy amplio en el área de las ciencias de la computación, llegándose aplicar en diversos sectores de la industria. Por otra parte, la visión computacional se encarga de reconocer y localizar objetos en un ambiente mediante el procesamiento de imágenes (\cite{image2}).


En la actualidad existen varias aplicaciones que usan visión computacional junto con inteligencia artificial, brindándoles la capacidad de identificar objetos que se puedan apreciar mediante una cámara o imágenes guardadas en el dispositivo. Un claro ejemplo es Huawei, que en sus últimos teléfonos agrego una nueva función a la aplicación de su cámara, que le permite detectar al objeto que se le tomará una fotografía, dependiendo del objeto, la aplicación usará diferentes efectos en la foto para que se logre visualizar mejor. Este proceso se puede observar en la Figura \ref{fig:huawei}, donde el teléfono decide que lo correcto es usar el efecto \textit{Bokeh} (desenfoque), también conocido como modo retrato.
\vspace{5mm}

\begin{figure}[h]
	\centering
	\includegraphics[width=7cm, height=5.1cm]{imagenes/huawei.png}
	\caption{Funcionamiento de la cámara de Huawei.}
	\source{Huawei (2018).}
	\label{fig:huawei}
\end{figure}
\clearpage
Los automóviles con piloto automático son otro ejemplo del uso de estas tecnologías, como ser el caso de los vehículos de Tesla (\cite{tes}). Mediante el uso de esta tecnología y una serie de cámaras pueden llegar a detectar diferentes objetos que lo rodean, esto se puede apreciar en la Figura \ref{fig:tesla}, dándoles así la  capacidad de poder reconocer señales de tránsito, personas y objetos, lo cual brinda la información necesaria para que el vehículo tenga una especie de piloto automático.

\begin{figure}[h]
	\centering
	\includegraphics[scale=0.22]{imagenes/tesla.jpg}
	\caption{Detección realizada por los vehículos Tesla.}
	\source{Tesla (2019).}
	\label{fig:tesla}
\end{figure}
En el sector de los supermercados no aparecieron grandes innovaciones en la ultima década para facilitar la atención, más allá de aplicar la teoría de colas. Con el paso del tiempo surgieron ideas como el uso del sistema de fila única y el método de auto cobro de productos, que consiste que el mismo cliente escanea sus compras y realiza el pago mediante tarjeta. Pero la innovación principal es con la llegada de la inteligencia artificial y visión computacional, las cuales se hicieron más accesibles, debido a eso surgieron nuevas soluciones, como ser el caso de las tiendas de Amazon, que no necesitan tener cajeros (\cite{ama}), ya que un sistema de camaras y unas etiquetas especiales en los productos permiten realizar todo el proceso de cobranza de los productos.
\\

Teniendo en cuenta estas últimas innovaciones, se quiere realizar una aplicación con visión computacional junto a una red neuronal, con el fin de poder detectar a las personas que estén esperando en una fila y así calcular el tiempo promedio de espera actual, este tiempo obtenido se comparará con el máximo establecido, si este llega a ser mayor se procederá en abrir una nueva caja. Para desarrollar esto se pretende utilizar redes ya construidas como YOLO, cuyo funcionamiento se puede observar en la Figura \ref{fig:deyolo}, o también considerar la opción de crear una propia red mediante el uso de diferentes herramientas como Tensorflow, Keras y Pytorch. La elección del método dependerá de las exigencias del proyecto y el lenguaje de programación a escoger.
 
\begin{figure}[h]
	\centering
	\includegraphics[scale=0.4]{imagenes/person_detection.png}
	\caption{Detección lograda por la red neuronal YOLO}
	\source{YOLO (2018).}
	\label{fig:deyolo}
\end{figure}

\vspace{5mm}

\section{Planteamiento del Problema}


Actualmente, las tiendas están muy conscientes sobre la importancia de mantener una experiencia placentera al momento de atender a un cliente, ya que un cliente bien atendido aumenta las probabilidades de que la persona decida volver al local y eso generaría más beneficios para tienda. De acuerdo con un estudio realizado por la revista \textit{Journal of  Business Research} (\cite{shoppingExp}), el  factor más influyente al momento de tomar una decisión son las filas largas que se generan, un ejemplo de esto se puede observar en la Figura \ref{fig:longF}.

\vspace{5mm}
 Una de las soluciones que normalmente se aplica es el generar una fila única con varias cajas (\cite{cola2323}), para así lograr una atención efectiva, pero el éxito de este método es saber cuantas cajas tendrán que estar disponibles para atender, porque teniendo todas las cajas abiertas en todo momento no es muy eficiente para el supermercado, ya que habría cajas vacías y eso ocasiona tener personal que no este realizando ninguna actividad, por otra parte, teniendo poco personal atendiendo las cajas ocasionaría largas filas y eso conlleva a generar una mala experiencia al cliente. 
 

\vspace{5mm}

\begin{figure}[h]
	\centering
	\includegraphics[scale=0.7]{imagenes/fila_larga.png}
	\caption{Representación de una fila larga.}
	\source{Shutterstock}
	\label{fig:longF}
\end{figure}

\vspace{5mm}



\section{Objetivos}


\subsection{Objetivo General}


Desarrollar una aplicación utilizando YOLO para detectar personas y así poder calcular el tiempo de espera promedio de los clientes en un sistema de atención de fila única en un supermercado.

\subsection{Objetivos Específicos}


\begin{itemize}
    \item Seleccionar el algoritmo de detección de objetos basado en los requerimientos del proyecto.
    \item Entrenar una red neuronal para que sea capaz de detectar personas a través de videos de cámara de seguridad.
    \item Realizar una aplicación que pueda efectuar un seguimiento de las personas en una fila de supermercado.
    \item Documentar el desarrollo de la aplicación y las herramientas utilizadas.
\end{itemize}

\vspace{10mm}

\section{Motivación}

En los últimos años la inteligencia artificial junto con la visión computacional son tecnologías que están siendo usadas en diversos campos en la industria, que van  desde el sector automotriz, financiero, educativo y hasta sistemas de seguridad. Tomando en cuenta esto, desarrollar aplicaciones que utilicen estas herramientas tendrán un gran impacto al mercado local, ya que no hay muchas soluciones que utilicen estas tecnologías y como ser algo nuevo seguirá creciendo hasta llegar a solucionar problemas mas cotidianos. Como en este caso al utilizar estas herramientas para solucionar un problema que surge en los supermercados al utilizar un sistema de fila única.\clearpage
\section{Justificación}

 Un tiempo de espera elevado afectan tanto al cliente (más tiempo de espera) como al supermercado (cajeros muy ocupados). Reducir la espera de un comprador habilitando más cajas para la atención, podría mejorar la experiencia de compra y con ello aumentar las probabilidades de retorno del cliente al local. Por tanto, este proyecto se lleva a cabo para ayudar a determinar cuántas cajas tienen que estar abiertas en un sistema de fila única.
 \\
 
Con esta información el supermercado puede determinar en qué momento necesitará mayor personal para mejorar la atención y cuándo no, con esto podrá tomar la decisión si es necesario reducir la cantidad de empleados en los periodos que no son necesitados.

\section{Delimitación}
Los límites y alcances del proyecto, se resumen en los siguientes puntos: 
\subsection{Límites}
El trabajo se desarrollará considerando las limitaciones impuestas por los equipos y cámaras de seguridad que se encuentran dentro del supermercado Makro Parque. Él mismo, está restringido por el sistema operativo que utilizan las computadoras para el monitoreo de seguridad ya que las tecnologías a utilizar tienen que ser compatibles con el sistema operativo Windows.



\subsection{Alcances}

En el presente trabajo se demostrarán los beneficios de utilizar la red neuronal YOLO para detección de personas, realizando la implementación de una aplicación que permitiría a los supermercados de la ciudad de Santa Cruz de la Sierra, poder reducir el cuello de botella que se genera a partir de la aglomeración de clientes en una fila. Esta aplicación solo podría solucionar este problema a supermercados que utilicen un sistema de fila única. 

















\chapter{MARCO TEÓRICO}
\section{Estado del Arte}
\subsection{Evolución de los Sistemas de filas de un supermercado}
El primer supermercado se inauguró en el año 1916 en Tennessee, Estados Unidos, con el nombre Pegly Wiggly, pero presentaba un sistema de cajas que se concentraban más en facilitar y agilizar el trabajo del cajero, sin darle mucha importancia al cliente, ya que las personas no tenían mucho espacio de maniobra, como se puede observar en la Figura \ref{fig:qw}.

\vspace{5mm}
\begin{figure}[h]
	\centering
	\includegraphics[scale=0.3]{imagenes/pegly.jpg}
	\caption{Sistema de caja del primer supermercado.}
	\source{Xataka (2016).}
	\label{fig:qw}
\end{figure}

La gran innovación surgió cuando se incorporó el concepto de la teoría de colas, que fue introducido por Agner Kraup Erlang,  cuyo articulo estaba enfocado en las llamadas telefónicas. 
Teniendo los conceptos ya establecidos, se fueron probando diferentes sistemas para agilizar la atención, ya que cada vez mas gente realizaba sus compras en las tiendas. Estos modelos son: FIFO, LIFO y SJF (\cite{fifo}). 
\clearpage

La búsqueda de mejoras para agilizar la atención en las cajas es un hecho, ya que estudios demuestran la importancia de generar una gran experiencia de compra, porque eso aumenta la probabilidad que el cliente vuelva al local a realizar otra compra, (\cite{enjoy1}); (\cite{enjoy2}); (\cite{shoppingExp}); (\cite{enjoy3}).\\ 

Una de las nuevas innovaciones viene dada por parte de Amazon, donde abrió locales en el que no existen cajas, como se observa en la Figura \ref{fig:amago}. Este nuevo sistema que utiliza Amazon realiza un cobro mediante el uso de  cámaras con visión computacional para así detectar que productos selecciona el cliente, esos productos seleccionados se agregan directamente a un carrito virtual en la aplicación de Amazon, el cobro se realiza cuando el cliente sale del local (\cite{ama}); (\cite{ama2}).

\vspace{5mm}
\begin{figure}[h]
	\centering
	\includegraphics[scale=0.3]{imagenes/amago.jpg}
	\caption{Supermercado de Amazon que no utiliza cajas.}
	\source{Alicia Davara (2019).}
	\label{fig:amago}
\end{figure}

Otra innovación viene de los supermercados de China, donde implementaron un sistema de pago mediante el uso de la aplicación WeChat, en este caso los clientes tienen que escanear los diferentes productos que compraran y esto genera una \say{lista de compra}, la cual al momento de realizar el pago se crea un código QR, que posteriormente es escaneado por el cajero para registrar la compra y asei realizar el pago a travez de WeChat (\cite{wechat}), de esta manera se agiliza la atención y se evita que se genere largas filas. 
\clearpage
\subsection{Innovación en las redes neuronales}

 El concepto de las redes neuronales artificiales fue introducido por primera vez en 1943 por el matemático Walter Pitts y el neurólogo Warren McCulloch, mediante un articulo con el nombre \say{\textit{A Logical Calculus of Ideas Immanent in Nervous Activity}} (\cite{neu1}), en el cual se plantea un modelo computacional simplificado de como las neuronas biológicas podrían trabajar juntas, para poder realizar cálculos complejos utilizando modelos de lógica proposicional (\cite{neural}).
 
 El siguiente salto viene de la mano del psicólogo  Frank Rosenblatt, que por el año 1960 introdujo un nuevo concepto conocido como \say{perceptron} (\cite{neur2}), que permitía usar las redes neuronales como un clasificador binario (\cite{neu3}). Al principio solo podía representar funciones \textit{AND} y \textit{OR} pero en los siguientes años se pudo combinar varias redes neuronales que permitían representar más funciones lógicas.
 \vspace{5mm}
 
El dato sustancial para este proyecto es la introducción de las redes neuronales convolucionales o CNN (\textit{Convolutional Neural Network}) por sus siglas en inglés. El concepto viene por la investigación de Yann LeCun que permite trabajar con imágenes (\cite{neu4}); (\cite{neu5}). La manera en la que se usa estas redes se observa en la Figura \ref{fig:gato}. Este estudio genero muchas posibilidades de mejorar diversos sectores de la industria.
 
 \vspace{5mm}
 \begin{figure}[h]
	\centering
	\includegraphics[scale=0.2]{imagenes/gato.png}
	\caption{Aplicación de redes neuronales convolucionales para la clasificación de imágenes.}
	\source{Hamilton Chang (2019).}
	\label{fig:gato}
\end{figure}
En el proceso de la   se pasa por diferentes procedimientos, uno de ellos es la multiplicación con un \textit{kernel}, este proceso se apreciará mejor en la Figura \ref{fig:conv24}. En la Figura \ref{fig:conv23} observamos el resultado final de la convolución, donde a las entradas de la primera y última fila, y la primera y última columna se les ha asignado el valor original (\cite{conv2323}).
\clearpage
 \begin{figure}[h]
	\centering
	\includegraphics[scale=0.4]{imagenes/convu2.png}
	\caption{Procedimiento de convolución.}
	\source{Modelling in Science Education and Learning
Volume 9 (2016).}
	\label{fig:conv24}
\end{figure}
 \begin{figure}[h]
	\centering
	\includegraphics[scale=0.4]{imagenes/convu3.png}
	\caption{Resultado de una convolución.}
	\source{Modelling in Science Education and Learning
Volume 9(2) (2016).}
	\label{fig:conv23}
\end{figure}
Las redes neuronales convolucionales son la base de la mayoría de los proyectos que utilicen reconocimiento de objetos en imágenes (\cite{neu6}); (\cite{neu7}); (\cite{neu8}); Estos proyectos van desde el reconocimiento facial, clasificación de imágenes, detección de letras manuscritas y otros más. Las mejoras de esta red están enfocadas en aumentar la velocidad de respuesta y hacer más accesibles esta tecnología a más dispositivos, ya que un problema al trabajar con videos en tiempo real es la latencia. Este problema se solucionará con YOLO, gracias a su manera de aplicar CNN (\cite{yolo}), como se muestra en la Figura \ref{fig:yo}, se obtiene unos tiempos de ejecución favorables a este nuevo método. Donde COCO es una base de datos y está medido en porcentaje de acierto.
 \clearpage
 \begin{figure}[h]
	\centering
	\includegraphics[scale=0.2]{imagenes/image.png}
	\caption{Prueba de velocidad de YOLO con la base de datos COCO.}
	\source{YOLO (2018).}
		% TODO: Cambiar el caption
	\label{fig:yo}
\end{figure}
 
 
\section{Fundamentos Teóricos}
\subsection{Redes neuronales artificiales}

Las redes neuronales artificiales siguen el mismo funcionamiento que las neuronas biológicas, en el que las dendritas se encargan de captar los impulsos nerviosos que emiten otras neuronas, luego se las procesa y si es necesario se envía a otra neurona. En el caso de las neuronas artificiales, la suma de las entradas multiplicadas por el valor asociado determina el \say{impulso nervioso} que recibe la neurona. Este valor se procesa en el interior de la célula mediante una función de activación que devuelve un valor que se envía como salida de la neurona, esto se apreciará mejor en la Figura \ref{fig:natu} (\cite{open}). 

\vspace{5mm}
\begin{figure}[h]
	\centering
	\includegraphics[scale=0.216]{imagenes/natural}
	\caption{Red neuronal artificial en comparación a una neurona.}
	\source{Magiquo (2019).}
	\label{fig:natu}
\end{figure}
\vspace{4mm}

La palabra red, en el término \say{red neuronal artificial} se refiere a las interconexiones entre las neuronas en las diferentes capas de cada sistema. Una configuración ejemplar tiene tres capas. La primera posee neuronas de entrada que envían datos a través de las sinapsis a la segunda capa, y luego a través de más sinapsis a la tercera capa de salida. Los sistemas más complejos tendrán mayor número capas. Como se observa en la Figura \ref{fig:redneu} se presenta un ejemplo de cómo una neurona artificial tiene diferentes capas y cómo se conectan entre ellas.

\vspace{5mm}
\begin{figure}[h]
	\centering
	\includegraphics[scale=0.35]{imagenes/ela1}
	\caption{Estructura de una red neuronal artificial.}
	\source{Elaboración propia.}
	\label{fig:redneu}
\end{figure}

\subsubsection{Capas}

Para empezar una red neuronal es un conjunto de capas, y una capa se puede considerar como un conjunto de neuronas, en el cual sólo serán contadas a partir de la primera capa que sería la entrada.
Como se observa en la Figura \ref{fig:capa}, esa red neuronal tiene 3 capas, que son: la de entrada, la oculta y la de salida, pero en realidad sólo habría dos capas porque se descarta la capa de entrada. A simple vista, uno podría notar que las capas envían múltiples datos a la siguiente pero no es el caso, ya que dependiendo de la situación sólo una neurona enviará la información a la siguiente sección y de esa sección solo en una pasara la información, este procedimiento se repite hasta llegar a la salida.

\begin{figure}[h]
	\centering
	\includegraphics[scale=0.5]{imagenes/capa.png}
	\caption{Diferentes capas conectadas de una red neuronal.}
	\source{Elaboración propia}
	\label{fig:capa}
\end{figure}

\subsubsection{Métodos de aprendizaje}


Dependiendo del problema a resolver, el método de aprendizaje puede variar, de esta manera se podrá adecuar al proyecto de una mejor forma, estos métodos se distinguen en tres tipos de esquemas:

\subsubsection{Aprendizaje supervisado}


En el aprendizaje supervisado, los diferentes algoritmos trabajan con datos \say{etiquetados} (\textit{Labeled data}), que intentan encontrar una función, que dada a las variables de entrada \textit{(input data)}, se puede llegar asignar una etiqueta adecuada como se observa en la Figura \ref{fig:AprenSup}. El algoritmo se entrena mediante una base de datos y puede \say{aprender} a asignar las etiquetas de salida adecuadas (\cite{AprendizajeSup}).
\vspace{5mm}
\\
El aprendizaje supervisado se suele usar en:
\begin{itemize}
    \item Problemas de clasificación (identificación de dígitos, diagnósticos, o detección de fraude de identidad).
    \item Problemas de regresión (predicciones meteorológicas, de expectativa de vida, de crecimiento, etc).
\end{itemize}

\vspace{5mm}
\begin{figure}[h]
	\centering
	\includegraphics[scale=0.7]{imagenes/aprendizaje_supervisado.png}
	\caption{Aprendizaje supervisado.}
	\source{Diego Calvo (2016).}
	\label{fig:AprenSup}
\end{figure}
\clearpage

Estos dos tipos principales de aprendizaje supervisado, clasificación y regresión, se distinguen por el tipo de variable objetivo. En los casos de clasificación, es de tipo categórico, mientras que, en los casos de regresión, la variable objetivo es de tipo numérico.

Los algoritmos mas habituales al usar aprendizaje supervisado son:
 \begin{itemize}
    \item Árboles de decisión.
    \item Clasificación de Naïve Bayes.
    \item Regresión por mínimos cuadrados.
    \item Regresión Logística.
    \item Support Vector Machines (SVM).
    \item Métodos \say{Ensemble} (Conjuntos de clasificadores).
 \end{itemize}

\subsubsection{Aprendizaje reforzado}
El objetivo del aprendizaje reforzado, consiste que el mismo sistema pueda aprender y tomar decisiones por su misma experiencia. Es decir, dependiendo a la situación determinada el sistema va a seleccionar la opción optima, para así ejecutar un proceso. Esta respuesta mejorará con el tiempo médiate un procedimiento de prueba y error, que continuará hasta alcanzar al objetivo deseado (\cite{refor}).

\vspace{5mm}

Para realizar este entrenamiento se necesita de un “agente” en un estado determinado dentro de un ambiente, esta configuración se observa en la Figura \ref{fig:aprRefor}. Este método necesita una recompensa cada vez que alcance un objetivo, también se le dará una recompensa cuando alcance escenarios no deseados.

\begin{figure}[h]
	\centering
	\includegraphics[scale=0.5]{imagenes/apreRefor.png}
	\caption{Esquema de aprendizaje reforzado.}
	\source{Tao Bain (2017).}
	\label{fig:aprRefor}
\end{figure}

\subsection{Función de activación}
Ésta es otra similitud que tiene una red neuronal biológica con una artificial, esta función de activación es considerada como un impulso nervioso que es enviado al cerebro. Esta función de activación utiliza una suma ponderada de la entrada anterior.
\\

Existen varias funciones de activación que actúan de diferente manera y eso nos permite adecuar las capas de las neuronas, estas funciones son (\cite{funAc}):
\subsubsection{Sigmoid/Sigmoide}

Esta función se encarga de transformar los valores introducidos a una escala de una décima (0,1), la Ecuación (\ref{eq:sigmoideq}) tomará los valores de alto rango y lo transformara de manera asintomática a 1 y los valores bajos tienden de manera contraria acercándose al 0 (\cite{activa2}), la gráfica de la función se apreciará en la Figura \ref{fig:sigmoid}.

\vspace{5mm}
\begin{equation}
\label{eq:sigmoideq}
	f(x)=\frac{1}{(1-e^{-x})}
\end{equation}
\vspace{5mm}

Esta función se caracteriza en los siguientes factores :
\begin{itemize}
	\item Posee una lenta convergencia.
	\item Tiene un gran rendimiento en la ultima capa.
	\item Tiene buena acotación entre el 0 y 1.
\end{itemize}
\vspace{15mm}
\begin{figure}[h]
	\centering
	\includegraphics[scale=0.5]{imagenes/save.png}
	\caption{Representación gráfica de la función Sigmoid.}
	\source{Elaboración propia.}
	\label{fig:sigmoid}
\end{figure}

\clearpage
\subsubsection{Tangente hiperbólica(Tanh)}

Esta función tangencial consiste en transformar los valores de entrada en una escala (-1,1), donde los valores altos tenderán de manera asintomática a 1 y los valores de bajo nivel tenderán a 0, (\cite{activa2}). La gráfica de esta función se representará en Figura \ref{fig:tange}, la cual se obtiene con la Ecuación (\ref{eq:tangente}).
\begin{equation}
\label{eq:tangente}
	f\left( x\right) =\frac {2}{1+e^{-2x}}-1
\end{equation}


Las características de esta función son las siguientes:

\begin{itemize}
	\item Presenta una lenta convergencia.
	\item Se utiliza para decisiones de una sola opción.
	\item Tiene un gran rendimiento en redes recurrentes.
\end{itemize}

\vspace{5mm}
\begin{figure}[h]
	\centering
	\includegraphics[scale=0.15]{imagenes/tange.png}
	\caption{Representación gráfica de la función Tangencial.}
	\source{Travail personnel (2008).}
	\label{fig:tange}
\end{figure}

\subsubsection{Relu}

Esta función consiste en transformar todos los valores negativos a 0 y mantiene solo los positivos. Esta función de activación es la más utilizada a pesar de que existen otras más recientes (\cite{activa2}). La representación gráfica de esta función se encuentra en la Figura \ref{fig:relu} y se lo obtiene mediante la Ecuación (\ref{eq:relus}).

\vspace{5mm}
\begin{equation}
\label{eq:relus}
	F\left( x\right) =max\left( 0,x\right) =\begin{cases}0 para <0\\
x para\geq 0\end{cases}
\end{equation}
\vspace{5mm}

Algunas de las características de esta función son las siguientes:

\begin{itemize}
	\item Solo se activará cuando los valores sean positivos.
	\item Se comporta bien con imágenes.
	\item Tiene un gran desempeño con las redes neuronales.
\end{itemize}

\begin{figure}[h]
	\centering
	\includegraphics[width=9cm, height=5.9cm]{imagenes/relu.png}
	\caption{Representación gráfica de la función Relu.}
	\source{Visweswaran N. (2020).}
	\label{fig:relu}
\end{figure}



\subsubsection{Softmax}

La función Softmax consiste en calcular las probabilidades de un evento sobre \say{n} eventos diferentes. Es decir que esta función calculara las probabilidades de cada clase de objeto sobre todas las diferentes clases de objetivos posibles. El rango de esta función será de (0,1), y todas las sumas de las probabilidades serán igual a uno. Esta función se utiliza más que todo para obtener la probabilidad de modelos de multi clasificación (\cite{activa2}). La representación gráfica de la función está en la Figura \ref{fig:soft} y se la obtiene con la Ecuación (\ref{eq:softmax}).

\vspace{5mm}
\begin{equation}
\label{eq:softmax}
	f\left( z\right) _{j}=\frac {e^{zj}}{\sum ^{k}_{k=1}e^{z}k}
\end{equation}
\vspace{5mm}

La función de \textit{Softmax} se caracteriza por las siguientes razones:
\begin{itemize}
	\item Se utiliza cuando se quiere una respuesta en probabilidades.
	\item Se ocupa cuando se desea normalizar un tipo multiclase.
\end{itemize}

\begin{figure}[h]
	\centering
	\includegraphics[width=9cm, height=7cm]{imagenes/Softmax.png}
	\caption{Representación gráfica de la función Softmax.}
	\source{Saimadhu Polamuri (2016).}
	\label{fig:soft}
\end{figure}

\subsection{Inteligencia artificial}
La Inteligencia Artificial o IA es la combinación de diferentes algoritmos que con el tiempo se espera alcanzar o simular las capacidades del ser humano. La inteligencia artificial se puede considerar uno de los más grandes avances de la última década, ya que tiene diversas aplicaciones en diferentes áreas, por ejemplo, en la industria de la música se utiliza la IA para poder identificar el estilo musical que le puede gustar a una persona, por otra parte en la industrial automotriz ya existe autos que tiene la capacidad de que se manejen solos, un claro ejemplo son los vehículos de Tesla que cada vez van agregando mejoras a su sistema.

\vspace{5mm}
Cuando se habla de inteligencia artificial entramos al sector de ciencias de computación, donde esta área abarca los conceptos de \textit{machine learning, data mining y data science}, esta división se observa en la Figura \ref{fig:ia}.

\vspace{5mm}
 \begin{figure}[h]
	\centering
	\includegraphics[width=12cm, height=7cm, ]{imagenes/ia.jpg}
	\caption{División de los sectores en ciencias de computación.}
	\source{Machine learning and IA for Healthcare (2019).}
	\label{fig:ia}
\end{figure}
\vspace{5mm}

El concepto de la inteligencia artificial fue introducido en el año 1950 por Alan Turing, quien fue el primero en construir una máquina con la capacidad de pensar, la máquina que construyo se observa en la Figura \ref{fig:alan}, que fue conocido como \say{La máquina Colossus}, que tenia como objetivo poder descifrar los códigos enigma que utilizaban el ejército alemán en sus comunicaciones. También gracias a él, se creó una prueba conocida como \textit{The Turing Test}, el cual consiste en medir la capacidad que tiene una máquina en demostrar su \say{inteligencia artificial}, la prueba consiste en establecer una comunicación entre dos personas y una máquina, donde una persona no sabrá quien es la máquina o si hay una, si la persona no llega identificar a la máquina la prueba se considera como exitosa, ya que fue capaz de imitar el comportamiento humano (\cite{ia2}). 

\vspace{5mm}
 \begin{figure}[h]
	\centering
	\includegraphics[scale=0.3]{imagenes/alan1.jpg}
	\caption{La máquina Colossus creada por Alan Turing.}
	\source{BatallasHistoricas (2016).}
	\label{fig:alan}
\end{figure}

\subsubsection{Visión computacional}

La visión computacional es un campo que va muy conectado con la inteligencia artificial, que consiste que una máquina tenga un alto nivel de comprensión de imágenes o videos. Con esta tecnología se quiere automatizar tareas que requieran un sistema visual, por ejemplo, sistemas de clasificación de imágenes. Ahora, lograr una interpretación de imágenes al mismo nivel que el ser humano es un problema complejo. Sin embargo, en los últimos años hubo avances considerables, uno de ellos son el uso de esta tecnología para el piloto automático de los automóviles.
\subsubsection{OpenCV}
Actualmente, una de las herramientas más usadas para el campo de visión computacional es OpenCV, ya que es libre de uso y presenta varias opciones para trabajar en diversas aplicaciones que van desde reconocimiento de objetos, calibración de cámaras, visión estereoscópica y visión robótica.

\vspace{5mm}

OpenCV presenta una estructura modular, la cual dependiendo de la versión puede variar de paquetes que lo compongan, pero los principales son los siguientes:

\begin{itemize}
	\item Procesamiento de imagen.
	\item Análisis de videos.
	\item Calibración de cámara y reconstrucción 3D.
	\item Detección de objetos.
	\item Video I/O.
	\item Núcleo funcional.
\end{itemize} 

\subsection{Machine Learning}

El aprendizaje por máquina o \textit{machine learning} en inglés, consiste en poder extraer información de una base de datos y poder aprender de ella. El uso de \textit{machine learning} en los últimos años se hizo muy común, desde las páginas web como ser Facebook, YouTube o Amazon poseen algún tipo de \textit{machine learning}, ya que dependiendo del contenido que uno mire o producto que se busque generará información para qué la página aprenda y sepa que tipo de contenido recomendar (\cite{machine}).


El problema más común para el cual se utiliza \textit{machine learning}, es para la toma de desiciones, tomando en cuenta ejemplos de casos anteriores. Con esa información la \say{máquina} puede predecir cúal sería la salida dependiendo la entrada que se le entregue, esta respuesta que entrega podría ser una nueva creada por sí misma para satisfacer a la \say{pregunta}, todo esto lo realiza sin la intervención de un individuo.
\subsubsection{Tensorflow}


Tensorflow, es una biblioteca de software de código abierto para la computación numérica, esta herramienta fue desarrollado por Google. Tensorflow se creó para poder construir y entrenar redes neuronales, las cuales permite detectar, descifrar patrones y correlaciones en los datos que se están estudiando. En los últimos años, el interés por esta biblioteca que creó Google fue creciendo tal como se muestra en la Figura \ref{fig:datosT}, ya que es una herramienta muy útil para proyectos que tengan el uso de \textit{Machine learning}.

\begin{figure}[h]
	\centering
	\includegraphics[scale=0.35]{imagenes/tensor2323.png}
	\caption{Interés de búsqueda sobre Tensorflow en Google.}
	\source{Elaboración propia.}
	\label{fig:datosT}
\end{figure}

La plataforma tuvo sus inicios en el 2011 por parte del equipo Google Brain, que su lema se aprecia en la Figura \ref{fig:google22}, la primera versión de la plataforma era conocido con el nombre de DistBelief. Por el año 2014 el proyecto fue creciendo y tomando forma, así que se le cambió el nombre y ahí fue que le conoció como Tensorflow. Google liberó este software en su segunda versión el 9 de noviembre del 2015. Al pasar el tiempo esta herramienta se convirtió el más usado en el campo de Deep Learning, y eso se debe por la filosofía de Google: «código primero, código siempre».

\vspace{5mm}
\begin{figure}[h]
	\centering
	\includegraphics[scale=0.4]{imagenes/googleTeam.png}
	\caption{Equipo que desarrollo Tensorflow.}
	\source{Google.}
	\label{fig:google22}
\end{figure}

Para el año 2016, Google anunció una nueva función con el nombre de TPU o unidad de procesamiento de tensor, el cual es una construcción de un circuito integrado de aplicación específica o ASIC por sus siglas en inglés. Este circuito permite realizar un aprendizaje automática y que se adapta para Tensorflow. Esta implementación del TPU es un acelerador programable, con el fin de conseguir un alto \textit{throughput} (número de mensajes recibidos con éxito), y para correr modelos más rápidos al generar un entrenamiento. 

\subsubsection{Gráficos de cómputo}

Usando Tensorflow nos permite crear algoritmos de aprendizaje que puedan interactuar entre sí, y estas interacciones se los llega a conocer como grafo computacional.
Al referirse a un grafo en este tema, se quiere decir a un conjunto de nodos conectados entre sí, estos nodos normalmente tienen entrada y salida al mismo tiempo, cada nodo tiene a dentro alguna función aritmética, desde las más simples como ser sumas y multiplicaciones hasta las más complejas (\cite{tensor1}).

Para la creación de un grafo se usará un ejemplo básico del libro \textit{Learning Tensorflow} (\cite{tensor1}), Para empezar, se importará la librería al editor mediante el primer comando que se aprecia en el Código \ref{55}, siguiendo con esto se crearan unos 3 nodos con diferente valor.

\vspace{5mm}
 \begin{lstlisting}[language=Python,caption={Ejemplos de los primeros pasos para crear nodos con Tensorflow.},captionpos=b,label=55]
	import tensorflow as tf
	a = tf.constant(5)
	b = tf.constant(2)
	c = tf.constant(3)
\end{lstlisting}

Para la siguiente etapa se creará otros 3 nodos pero esta vez tendrán ecuaciones que se ejecuten dentro ellas y además uno de esos nodos tendrá una variable que no se definió anteriormente, la escritura del código se podrá apreciar en el Código \ref{56}.

 \begin{lstlisting}[language=Python,caption={Ejecución de funciones aritméticas en Tensorflow.},captionpos=b,label=56]
	d = tf.multiply(a,b)
	e = tf.add(c,b)
	f = tf.subtract(d,e)
\end{lstlisting}


La representación gráfica de las funciones creadas mediante código se podrán ver en la Figura \ref{fig:capa3}, cómo se puede observar los nodos se conectan entre sí y por ellos mismos crean otro nodo resultante de la operación del nodo \say{e} con el \say{d}, para así crear el nodo \say{f}. 

\begin{figure}[h]
	\centering
	\includegraphics[scale=1]{imagenes/capa3.png}
	\caption{Ejecución de funciones aritméticas en Tensorflow.}
	\source{Learning Tensorflow (2017).}
	\label{fig:capa3}
\end{figure}

\subsubsection{Keras}

Es una extensión de alto nivel de TensorFlow que permite construir y entrenar modelos de aprendizaje profundo, esta modificación hace que TensorFlow sea más fácil de usar sin sacrificar la flexibilidad y el rendimiento. Keras contiene varias implementaciones de los bloques constructivos de las redes neuronales como por ejemplo los \textit{layers}, función de objetivo y función de activación.

Algunas ventajas al usar esta herramienta son las siguientes:
\begin{itemize}
    \item Keras presenta una interfaz simple y optimizada para diferentes casos de uso para el usuario, ya que proporciona información clara sobre errores que se podrían presentar al correr algún programa.
    \item Los modelos que se pueden crear con Keras son mediante bloques de conexión que se pueden configurar con pocas restricciones.
    \item Presenta una gran facilidad al momento de crear bloques, capas, métricas y diferentes funciones.
 \end{itemize}

\subsubsection{Configuración de capas iniciales}

Uno de los elementos principales al momento de crear una red neuronal son las capas que lo componen, éstas tienen como función de extraer la información del conjunto de datos que se están utilizando. Hay que tomar en cuenta que esa información extraída no siempre será útil para el problema que se quiere solucionar.


Al crear estas capas con Keras se necesita iniciar con el siguiente comando “tf.keras.layers.Flatten” si se quiere trabajar con análisis de imágenes, que consiste en transformar una imagen en un formato adecuado para que se pueda trabajar, con esto quiero decir que por ejemplo teniendo una imagen de un tamaño de 75x75 pixeles, al ejecutar el comando esta imagen se transformaría en 5625 pixeles (75 * 75 = 5625). Esta acción de “aplanar” los píxeles es el trabajo de la primera capa que se utiliza, con esta capa no comenzara con él aprendizaje sólo adecuará los sets de datos.

\vspace{5mm}

Para la siguiente etapa se tiene que aplicar una nueva capa con 2 diferentes configuraciones, para la primera capa se definirá con 128 nodos o neuronas y con la función de activación “relu“, par la siguiente capa solo tendrá 10 nodos y una función de activación “softmax”, estos comandos se observarán en el Código \ref{57}.

 \begin{lstlisting}[language=Python,caption={Configuración con Keras para trabajar con imágenes.},captionpos=b,label=57]
model = keras.Sequential([
	keras.layers.Flatten(input_shape=(75,75)),
	keras.layers.Dense(128, activation='relu'),
	keras.layers.Dense(10, activation='softmax')
])
\end{lstlisting}

\subsection{Deep Learning}

Como su nombre indica \textit{Deep learning} es la parte profunda del aprendizaje de una inteligencia artificial, esto se entenderá mejor en la observando la Figura \ref{fig:deep1}. Entre mayor sea la cantidad de capas relacionado a un modelo de se lo conoce como \textit{\say{deep}} (\cite{deep23}).
\clearpage
 \begin{figure}[h]
	\centering
	\includegraphics[scale=1]{imagenes/deep2.jpg}
	\caption{Posición de Deep Learning en una inteligencia artificial.}
	\source{Machine learning and IA for Healthcare (2019).}
	\label{fig:deep1}
\end{figure}

Los datos de entrada que almacena una capa se lo conocen como \say{weights}, que en esencia son un montón de números. En términos más técnicos \say{weights} se lo conoce como parámetros de una capa.

\vspace{5mm}

 Así que este aprendizaje consiste en encontrar valores que satisfagan a los parámetros establecido en el \say{weights} de la red neuronal. 
 Pero estos parámetros pueden ser mas de  un millón, así que encontrar un valor que satisfaga a cada parámetro puede parecer tedioso, más aún que cuando se encuentra un valor eso afecta a los demás.


\subsection{YOLO}

YOLO (\textit{You Only Look Once}, \say{sólo se ve una vez}) es un algoritmo de visión artificial que detecta y clasifica objetos en tiempo real, que permite detectar diferentes objetos que se pueden presentar en una imagen y poder enmarcarlos con un cuadro alrededor de los objetos encontrados Figura \ref{fig:yolo1}.

\begin{figure}[h]
	\centering
	\includegraphics[scale=0.8]{imagenes/yolo1.png}
	\caption{Detección múltiple realizada por YOLO.}
	\source{YOLO (2018).}
	\label{fig:yolo1}
\end{figure}

La principal innovación que YOLO trajo consigo a este campo, fue el hecho de que es capaz de realizar múltiples detecciones de una sola vez, lo cual realiza este proceso de manera rápida y eficiente.
El procesamiento de imágenes con YOLO es simple y directo, por ejemplo, teniendo una imagen de entrada de un tamaño de 448x448ppx, esto se aprecia en la Figura \ref{fig:yolo2}, ejecuta una red convolucional a la imagen que llegaría a cambiar de tamaño y así limitar los resultantes de las detecciones confiando en el modelo que se usó para entrenar la red neuronal. 
\begin{figure}[h]
	\centering
	\includegraphics[scale=0.8]{imagenes/yolo2.png}
	\caption{Aplicación de YOLO en una imagen.}
	\source{YOLO (2018).}
	\label{fig:yolo2}
\end{figure}

Entrando más a detalle, YOLO primeramente empieza a cuadricular la imagen que se está usando para la detección, seguido de eso va dibujando unas \say{cajas delimitadoras} mediante la guía del modelo entrenado, este modelo puede tener diversas etiquetas para la identificación de objetos y determinará cuando lleguen a un porcentaje de confianza definido, todos esos pasos se pueden apreciar en la Figura (\ref{fig:yolo4}). 

 \begin{figure}[h]
	\centering
	\includegraphics[scale=0.6]{imagenes/yolo3.png}
	\caption{Proceso que realiza YOLO para generar una detección de uno o varios objetos dentro de una imagen.}
	\source{YOLO (2018).}
	\label{fig:yolo4}
\end{figure}

La estructura de la red neuronal YOLO esta basada en GoogLeNet, la cual es otra red neuronal que utiliza la estructura \textit{Inception}. Este tipo de estructura permite hacer uso de multiples filtros de diversos tamaños en una sola capa, con esto permite que la red se adapte mejor a los diversos casos de detección \cite{googlenet}.

Internamente YOLO está compuesto por 24 capas convolucionales y por 2 capas que se encuentran completamente conectadas, La estructura completa se puede observar en la Figura \ref{fig:yolo108}.


 \begin{figure}[h]
	\centering
	\includegraphics[scale=0.3]{imagenes/yolo-interno.png}
	\caption{Estructura interna de la red neuronal YOLO.}
	\source{YOLO (2016).}
	\label{fig:yolo108}
\end{figure}

Esta estructura por tener una gran cantidad de capas y filtros posee una gran detección, pero a un costo bastante en el rendimiento, sobre todo cuando se utiliza en computadoras que no poseen algún procesamiento gráfico (GPU), debido a esto surgen otras versiones de YOLO.


\subsubsection{Tiny Yolo}

Tiny Yolo es una versión reducida de YOLO, ya que su estructura esta compuesta por 7 capas convolucionales y 6 capas con la función \textit{max-pooling}, estas últimas capas tiene la función de resaltar los valores máximos encontrados en la imagen (\cite{tiny2}).

Al tener menos capas está red neuronal es capaz de tener un mejor rendimiento en computadoras que no cuenten con una GPU, pero por eso mismo la red pierde precisión al momento de detectar a objetos. La estructura de esta red neuronal se observa en la Figura \ref{fig:yolo109}.

 \begin{figure}[h]
	\centering
	\includegraphics[width=13cm, height=5.2cm]{imagenes/tiny-yolo.jpg}
	\caption{Estructura interna de la red neuronal Tiny YOLO.}
	\source{Wei Fang (2019).}
	\label{fig:yolo109}
\end{figure}

\subsubsection{Darknet}

Darknet es un \textit{framework} de \textit{open source} (código abierto) que permite trabajar con la red neuronal YOLO, este \textit{framework} esta escrita en C/CUDA. La finalidad de utilizar Darknet es poder entrenar la red neuronal para que tenga la capacidad de detectar personas y con ello poder utilizarlo en el proyecto.
\subsubsection{Darkflow}

Darkflow es una herramienta muy similar a Darknet, pero con la diferencia de que Darkflow utiliza Tensorflow como motor para la red neuronal, lo cual provoca que para usar esta red se tiene que realizar en un entorno que reconozca el lenguaje Python.

 


\subsection{Servicios de cómputo en la nube}

Computación en la nube es un término que normalmente se utiliza para denominar cualquier servicio que realice algún tipo de trabajo a través de internet. Para contratar un servicio de este estilo se tiene que hacer por AWS (\textit{Amazon Web Service}), Microsft Azure o Google Cloud.



\subsubsection{Amazon Web Service}

\textit{Amazon Web Services} o AWS en sus siglas en inglés, es un servicio en la nube que ofrece más de 175 servicios de manejo de datos a nivel global, estos servicios se pueden observar en la Figura \ref{fig:aws}. Ofrece tecnologías desde infraestructura para almacenamiento, servicios de cómputo hasta nuevas tecnologías como ser el aprendizaje automático e inteligencia artificial. 

 \begin{figure}[h]
	\centering
	\includegraphics[width=12cm, height=7cm, ]{imagenes/ama23.png}
	\caption{Servicio que ofrece AWS.}
	\source{Amazon (2019).}
	\label{fig:aws}
\end{figure}


Ofrece tres tipos de servicios principales en la nube, el cual son: infraestructura como servicio, plataforma como servicio y software como servicio. Cada tipo de servicio tiene diferentes niveles de control.

\vspace{5mm}

Se utilizó este servicio para crear una máquina en la nube que tenga las capacidades para correr el entrenamiento y que éste pueda estar encendido las 24 horas al día o por lo menos hasta que finalice el entrenamiento.

\vspace{5mm}

Para ser más específico el tipo de servicio que se utilizó es \textit{Amazon Elastic Compute Cloud} o EC2 por sus siglas en inglés, el cual es un servicio web que ofrece la capacidad de realizar cálculos de cómputo adaptable, que en otras palabras se refiere a hospedar sistemas de \textit{sotfware} para la ejecución de aplicaciones. 


\section{Google Colab}

Google Colab es una plataforma de programación a través de la nube, la ventaja de este servicio es que ofrece el uso de GPU para el entrenamiento de redes neuronales de manera gratuita, pero sólo se le puede dejar 5 horas en \textit{stand by} antes de que se cierre. Para usar este servicio lo único que se necesita es una cuenta en Google e iniciar sesión.

\section{Conclusiones del Capitulo II}

En el presente capítulo se presentaron los fundamentos teóricos relacionados con redes neuronales convolucionales y la inteligencia artificial. Los temas que se abordaron son las diferentes maneras en la que se puede crear una red neuronal y estructura que maneja YOLO. También se explica el uso de herramientas en la nube, la cual pueden facilitar el proceso de entrenamiento de una red neuronal.





















\chapter{MARCO PRÁCTICO}
\section{Elección de una red neuronal}
Para la elección de la red neuronal se tomó en cuenta un estudio realizado por el equipo de YOLO (\cite{persona}), este equipo realizo pruebas con diferentes redes neuronales para poder demostrar el gran rendimiento que tiene la red YOLO, estas pruebas fueron hechas en la misma computadora, el resultado de estas pruebas se puede observar en la Figura \ref{fig:labeling3} que da como ganador la red neuronal YOLO.

\begin{figure}[h]
	\centering
	\includegraphics[scale=0.6]{imagenes/persona9.png}
	\caption{Comparación de YOLO y otros detectores de objetos de última generación.}
	\source{YOLOv4: Optimal Speed and Accuracy of Object Detection. (2020)}
	\label{fig:labeling3}
\end{figure}

Estas pruebas fueron realizadas haciendo el uso de la base de datos \say{coco}, que recopila una gran información que facilita la detección de diferentes objetos. Con estos resultados nos muestran que la red neuronal YOLO es la mejor opción, ya que cuenta con una gran velocidad sin perder mucha precisión al momento de detectar los diferentes objetos. Otra red que tiene resultados muy buenos es YOLOv3, es algo lento pero tiene una gran precisión y dependiendo de la aplicación podría ser una buena elección.

\section{Entrenamiento de la red neuronal}
En este capítulo se ira desarrollando los diferentes pasos para lograr un buen entrenamiento de una red neuronal, en este caso el objetivo del entrenamiento es que sea capaz de detectar personas.

 \subsection{Base de datos}
 Para lograr el entrenamiento de la red neuronal, primeramente se tiene que construir una base de datos, esta base tiene que estar compuesta por imágenes donde aparezcan personas. Posteriormente, comienza un proceso de \say{etiquetado}, en el cual se tiene que señalar donde se encuentra la persona en las diferentes imágenes, este procedimiento se realiza con la aplicación LabelImg, este proceso se observa en la Figura \ref{fig:labeling}.
\begin{figure}[h]
	\centering
	\includegraphics[scale=0.25]{imagenes/persona1.png}
	\caption{Proceso de etiquetado para generar los archivos necesarios para el entrenamiento.}
	\source{Elaboración propia}
	\label{fig:labeling}
\end{figure}

Este proceso se tiene que realizar con cada imagen, para así obtener un archivo que especifica la posición de dónde se encuentra la \say{persona} en la imagen. En la Figura \ref{fig:labeling2} se observa la manera en la que se guardan esos datos. De esta manera se obtiene una base de datos que sirve para entrenar la red neruonal YOLO, la cual debe estar compuesta por lo menos con 400 imágenes. 

\clearpage

\begin{figure}[h]
	\centering
	\includegraphics[scale=0.3]{imagenes/persona2.png}
	\caption{Coordenadas obtenidas por el programa LabelImg al etiquetar a un objeto.}
	\source{Elaboración propia}
	\label{fig:labeling2}
\end{figure}

Este procedimiento se puede realizar mas rápido gracias a la base de datos de Google, el cual tiene una gran cantidad de imágenes ya etiquetadas lo que facilita el proceso de realizar un entrenamiento, esto se observa mejor en la Figura \ref{fig:labeling5}. El único inconveniente de esta herramienta es que la base de datos esta en un formato que Darknet no lo reconoce, pero ya existen otras herramientas que permiten convertir a un formato con la cual se puede trabajar.

\begin{figure}[h]
	\centering
	\includegraphics[scale=0.2]{imagenes/persona5.png}
	\caption{Base de datos de Google para el entrenamiento de redes neuronales.}
	\source{Elaboración propia}
	\label{fig:labeling5}
	\end{figure}

\subsection{Preparación del ambiente}

Al momento de entrenar una red neuronal se utiliza muchos recursos de la computadora, sobre todo en el apartado del GPU, ya que la mayoría de los cálculos se realizan de mejor manera en esa parte. Esto resulto ser un problema ya que mi computadora no cuenta con una GPU dedicada y los entrenamientos se realizaban de manera muy lenta, por ese motivo se opto por usar Colab, el cual es un servicio de Google que te permite realizar entrenamientos de redes neuronales ya que te deja usar los GPU's de sus servidores.
\\

Colab es un editor de texto online, que te permite escribir tu código y probarlo sin la necesidad de instalar programas en tu computadora, pero será necesarios instalarlos en tu editor online. El primer paso es instalar Darknet, el cual te permite trabajar con YOLO.

\clearpage

Para qué Darknet funcione correctamente necesita diferentes frameworks, en la Figura \ref{fig:labeling42} se puede observa una lista con los frameworks necesarios. Para poder instalarlos se utiliza el comando \say{pip}, el cual es un sistema de gestión de paquetes de Python. La mayoría de estos frameworks ya vienen instalados por defecto en Colab por ser muy utilizados.
 
\begin{figure}[h]
	\centering
	\includegraphics[scale=0.5]{imagenes/persona3.png}
	\caption{Frameworks necesarios para que funcione Darknet.}
	\source{Elaboración propia}
	\label{fig:labeling42}
\end{figure}

\subsection{Personalizar una red neuronal}

La red neuronal YOLO y Tiny-Yolo viene por defecto entrenada con varias clases que le permite detectar diferentes objetos, pero ninguna se podía ajustar a las necesidades del proyecto y por eso mismo se tuvo que realizar un entrenamiento con una base de datos propia.

Antes de modificar los paramatros, se decidio usar la Tyin-Yolo debido a que es una red neuronal mas liviana (con menos capas), lo que ocaciones que sea menos exigente cuando se tiene que procesar las detecciones, lo que permite que funcione de mejor manera en mi computadora.

La diferencia de rendimiento se podrá ver de mejor manera en la Tabla \ref{tab:fruta}, el cual da cómo mejor opción Tiny-Yolo teniendo un mejor rendimiento que al usar YOLO. El único problema de usar Tiny-Yolo es que se pierde un poco de precisión al momento de realizar las detecciones.

\begin{table}[h]
\caption{Resultados de una prueba de rendimiento entre YOLO y Tiny-Yolo.}
\begin{center}
%\caption{Resultados de una prueba de rendimiento entre YOLO y Tyni-Yolo.}
\begin{tabular}{| l | c |}
\hline \textbf{Versión de YOLO} & \textbf{FPS} \\ \hline
Tiny-yolo & 7-10  \\
Yolo & 1-3 \\ \hline
\end{tabular}

\label{tab:fruta}
\sources{Elaboración propia.}
\end{center}
\end{table}

Estos resultados demuestran que la mejor opción es trabajar con Tiny-Yolo, por lo cual se modificara el archivo \say{yolov4-tiny.cfg}, en el cual se cambiaran ciertos parámetros para que la red neuronal al momento de entrenar se adapte a los requerimientos del proyecto. los parámetros que se modificaron están señaladas en la Figura \ref{fig:labeling6}.

\begin{figure}[h]
	\centering
	\includegraphics[scale=0.4]{imagenes/persona6.png}
	\caption{Se define el learning rate.}
	\source{Elaboración propia}
	\label{fig:labeling6}
\end{figure}

El dato marcado \textit{learning rate} es el que determina el ritmo de aprendizaje, entre mas bajo sea este valor aumentara el ritmo de aprendizaje pero esto alarga el entrenamiento. Un valor muy bajo no es muy útil ya que no mejoraría el entrenamiento pero si demoraría mas tiempo en terminar y aumenta la probabilidad de que se detenga el entrenamiento por un error.
El siguiente es \textit{max batches}, el que se encarga de limitar el número de pasos máximos que realizará en el entrenamiento. Este número es obtenido mediante una simple fórmula que se observa en la Ecuación \ref{eq:maxb}, pero solo es aplicada cuando se trabajará con 3 o más clases de detección, caso contrario simplemente se tendrá que colocar \say{6000}, como es en este caso. En \textit{step} se tomará el 80\% y 90\% de \say{max batches} respectivamente.

\begin{equation}
\label{eq:maxb}
	max\_batches = \#\operatorname{clases} * 2000
\end{equation}

Otros parametros que se tienen que modificar son los datos marcadsos que aparecen en la Figura \ref{fig:labeling7}, que consiste en definir la cantidad de filtros y especificar cuántas clases tendrá que aprender la red neuronal.

\begin{figure}[h]
	\centering
	\includegraphics[scale=0.4]{imagenes/persona7.png}
	\caption{Se establece la cantidad de filtros a usarse.}
	\source{Elaboración propia}
	\label{fig:labeling7}
\end{figure}

Para calcular la cantidad de filtros se hará uso de una fórmula simple que se puede apreciar en la Ecuación \ref{eq:clases}, este dato obtenido se tendrá cambiar en todos los filtros que se encuentra en el documento \say{yolov4-tiny.cfg}.

\begin{equation}
	filtros = (\#\operatorname{clases}+5)*3
	\label{eq:clases}
\end{equation}

Con esto se finaliza los cambios en el archivo \say{yolov4-tiny.cfg}, pero aún falta editar otros 2 archivos más el cual uno definirá el nombre de la clase y en el otro se definirán la ubicación de los archivos necesarios para entrenar, la locación de donde se guardara la red entrenada y cuantas clases se usara para entrenar, esto se aprecia en la Figura \ref{fig:labeling45}.

\begin{figure}[h]
	\centering
	\includegraphics[scale=0.4]{imagenes/persona8.png}
	\caption{Se detalla la cantidad de clases y nombres que se usarán para entrenar la red neuronal.}
	\source{Elaboración propia}
	\label{fig:labeling45}
\end{figure}

\subsection{Inicio de entrenamiento}

Para iniciar el entrenamiento se tiene que preparar todos los archivos en el ambiente Colab, primeramente, se tiene que cargar todas las imágenes ya etiquetadas y de ahí se procedería a ejecutar los siguientes comandos que se observan en la Figura \ref{fig:labeling46}, el cual creará unos archivos basados con la cantidad de clases por el cual se entrenara.

\begin{figure}[h]
	\centering
	\includegraphics[scale=0.4]{imagenes/persona11.png}
	\caption{Comando para generar archivos que guiaran el entrenamiento.}
	\source{Elaboración propia}
	\label{fig:labeling46}
\end{figure}

Una vez teniendo los archivos generados se procede a iniciar el entrenamiento mediante el siguiente comando que aparece en la Figura \ref{fig:labeling47}. La duración de este entrenamiento dependerá de la cantidad de imágenes que se esta usando para realizar el entrenamiento, en este caso el entrenamiento duro aproximadamente 4 horas, pero se repitió 2 veces para mejorar los resultados obtenidos.

\begin{figure}[h]
	\centering
	\includegraphics[scale=0.5]{imagenes/persona12.png}
	\caption{Comando para iniciar el entrenamiento.}
	\source{Elaboración propia}
	\label{fig:labeling47}
\end{figure}

Al finalizar el entrenamiento ya se obtendrá una red neuronal funcional, capaz de detectar personas. Un resultado de este entrenamiento se observa en la Figura \ref{fig:labeling48}, donde se ven marcadas una cantidad aceptable de personas que aparecen en la imagen, cada recuadro que marca a una personas viene con un numero el cual indica el porcentaje de seguridad de detención realizada por la red neuronal.

\begin{figure}[h]
	\centering
	\includegraphics[scale=0.4]{imagenes/persona14.png}
	\caption{Resultado del entrenamiento realizado.}
	\source{Elaboración propia}
	\label{fig:labeling48}
\end{figure}

\section{Prueba de rendimiento}

Una vez entrenada la red neuronal se procede a realizar unas diversas pruebas para así poder calcular el rendimiento y sobre todo la precisión de esta red construida. Para estas pruebas se está usando la documentación que proporciona la pagina de desarrolladores de Google, donde se detalla la manera en la que se tiene que realizar estas pruebas para calcular la precision.

Para estas pruebas se está usando un video de vigilancia que fue proporcionado por el supermercado Makro, la posición de la cámara permite capturar bien la fila que se genera en el supermercado. Este video será analizado por la red neuronal y así calcular cuantas personas fueron detectadas para así realizar una comprobación que permita confirmar si la cantidad de personas detectadas es la correcta.  

Una vez procesado el video por la red neuronal se extrajo 44 imágenes, con estas imágenes se procedió a realizar el análisis el cual permitiría medir la precisión de esta red neuronal. Para ellos las diferentes detecciones se dividió en 3 diferentes categorías, estas categorías son: verdadero positivo (VP), falso positivo (FP) y falso negativo (FN).



\begin{figure}[h]
	\centering
	\includegraphics[scale=0.4]{imagenes/VPFPFN.png}
	\caption{Pantalla principal de la aplicación NeoVision.}
	\source{Elaboración propia}
	\label{fig:labeling30}
\end{figure}

\section{Diseño de la interfaz}
Para el diseño de la interfaz se utilizo la herramienta Balsamiq Wireframes, el cual permite planificar cómo se vera la interfaz de alguna aplicación. Con esta herramienta se diseño 2 pantallas para así determinar el diseño que tendrá la aplicación.

\subsection{Detección de personas}

Esta es la pantalla principal de la aplicación, donde se  indica la información de cuántas personas están siendo detectadas en ese momento, de igual manera se muestra en tiempo real el video obtenido por la cámara de seguridad. Otra información importante que se indica en la interfaz es el limite de detección, el cual ayuda a visualizar mejor los datos de esta pantalla, esto se apreciará mejor en la Figura \ref{fig:labeling30}. 

\clearpage
\begin{figure}[h]
	\centering
	\includegraphics[scale=0.4]{imagenes/imagen1.png}
	\caption{Pantalla principal de la aplicación NeoVision.}
	\source{Elaboración propia}
	\label{fig:labeling30}
\end{figure}

\subsection{Pantalla de alerta}

Esto es una pantalla de alerta que solo aparecerá cuando se pase el limite de tiempo de espera de una persona en la fila del supermercado. Esta pantalla tendrá una imagen de alerta para así poder captar la atención de las personas y se aconsejara abrir una nueva caja para poder acelerar el proceso de atención, el diseño de esta pantalla se puede ver en la Figura \ref{fig:labeling31}.


\begin{figure}[h]
	\centering
	\includegraphics[scale=0.4]{imagenes/imagen2.png}
	\caption{Mockup de la pantalla de alerta.}
	\source{Elaboración propia}
	\label{fig:labeling31}
\end{figure}

\section{Diagrama de flujo}
En este apartado se mostrará un diagrama de flujo en el que se representa los principales procesos de este proyecto, cuáles son detectar a personas y dependiendo del tiempo de espera notificar con una alerta, este diagrama se observa en la Figura \ref{fig:labeling33}.

\begin{figure}[h]
	\centering
	\includegraphics[scale=1]{imagenes/NeoVision3.png}
	\caption{Diagrama de flujo explicativo sobre la función de la aplicación NeoVision.}
	\source{Elaboración propia}
	\label{fig:labeling33}
\end{figure}

\clearpage
\section{Diagrama C4Model}
En este apartado se dará conocer un esquema siguiendo el modelo C4Model, el cual se va explicando de manera general hasta llegar a lo más específico, empezando por el contexto de la situación, sus contenedores, componentes y por último un diagrama de todas las clases.

\subsection{Contexto}
En este apartado se observará las diferentes interacciones que tiene la aplicación, esto se apreciará mejor en la Figura \ref{fig:labeling32}.


\begin{figure}[h]
	\centering
	\includegraphics[scale=0.5]{imagenes/contenido2.png}
	\caption{Diagrama de contexto que representa las interacciones del sistema.}
	\source{Elaboración propia}
	\label{fig:labeling32}
\end{figure}
Con este diagrama se puede mostrar cuáles son las interacciones que realiza la aplicación, como ser el personal de sistemas es el que inicia la aplicación y si es necesario configura ciertos parámetros, por otra parte, la aplicación interactúa con el personal de seguridad mediante avisos que ejecutara cuando el tiempo de espera de una persona sobrepase el tiempo configurado.
\subsection{Contenedores}

En esta sección se observa más a detalle los sistemas usados para que la aplicación funcione, esto se ve en la Figura \ref{fig:labeling34}.

\begin{figure}[h]
	\centering
	\includegraphics[scale=0.5]{imagenes/contenedor3.png}
	\caption{Diagrama de contenedores que representa las diferentes funciones de los sistemas.}
	\source{Elaboración propia}
	\label{fig:labeling34}
\end{figure}

Con este diagrama conocemos cuáles son las funciones de los diferentes sistemas que tiene la aplicación, por ejemplo, Darknet es el encargado de realizar las detecciones en las imágenes que le envía la aplicación principal. Por otra parte, Deepsort es el que recibe los resultados de Darknet y este aplica un ID único a cada detección, esta información es recibida por NeoVision, la cual le permite realizar los cálculos para determinar si es necesario abrir una nueva caja. 

\section{Descripción de la aplicación}



Para comenzar el desarrollo de la aplicación primeramente se tiene que crear una ambiente virtual de Python en la computadora, ya que se necesita instalar ciertos paquetes que podrían entrar en conflicto con algunas aplicaciones ya instaladas y también genera un mayor orden.
La herramienta ideal para realizar esto es VirtualenvWrapper, ya que es fácil de instalar y sobre todo es simple de ingresar al ambiente virtual, sobre todo cuando se compara con otras herramientas. Esta diferencia aprecia mejor en la Figura \ref{fig:labeling9}.

\clearpage
\begin{figure}[h]
	\centering
	\includegraphics[scale=0.5]{imagenes/persona15.png}
	\caption{Diferencia entre iniciar un ambiente virtual de Python entre VirtualWrapper y Virtualenv.}
	\source{Elaboración propia}
	\label{fig:labeling9}
\end{figure}


Con el ambiente virtual se procede a instalar los diversos paquetes necesarios, donde los principales son Tensorflow y OpenCV. Teniendo estos recursos instalados, se procede a convertir la red neuronal entrenada que se encuentra en un formato \say{.weights}, a un formato que Tensorflow soporte. Para ello usaremos un código disponible hecho por la misma comunidad, que está hecho en Python, en la Figura \ref{fig:labeling10} se muestra el comando para convertir la red.

\begin{figure}[h]
	\centering
	\includegraphics[scale=0.7]{imagenes/convertir_yolo.png}
	\caption{Comando para convertir la red neuronal.}
	\source{Elaboración propia}
	\label{fig:labeling10}
\end{figure}

 Para realizar la aplicación se hará uso del editor PyCharm, porque es gratis y tiene una interfaz intuitiva al momento de usarlo.
 El primer paso al comenzar la aplicación es comenzar a importar todos los paquetes que se necesitará para que funcione, la manera en la que se importa los paquetes se podrá observar en el siguiente Código \ref{23}.
 \vspace{5mm}

\begin{lstlisting}[language=Python,caption={Manera en la que se importan los diferentes paquetes al programa.},captionpos=b,label=23]
	import os 
	import subprocess
	from datatime import datetime
	import tensorflow as tf
	import time
\end{lstlisting}
\vspace{5mm}

Con todo esto ya se puede dar comienzo para que la aplicación pueda leer los videos mediante OpenCV y así iniciar el detector para que encuentre las personas de un video.

\subsection{Agregar el tracker a la detección}

Por sí solas la detección que se puede hacer no brinda la información suficiente como para contar las detecciones o corregir errores que se pueden generar. Uno de estos errores es que como en un video hay objetos en movimiento simplemente puede desaparecer las detecciones, debido a que por el movimiento no se llega a reconocer al objeto, esto se puede ver en la Figura \ref{fig:labeling12}.
\clearpage
\begin{figure}[h]
	\centering
	\includegraphics[scale=0.3]{imagenes/deteccionvstracker.png}
	\caption{Comparación entre usar solo la detección y usar tracker con detección.}
	\source{Elaboración propia.}
	\label{fig:labeling12}
\end{figure}

Cómo se ve en la figura anterior, no se marca todos los objetos en el lado que no utiliza \textit{tracker}, ya que justo en ese \textit{frame} no lo identifica bien y se pierde. Pero en otra parte del mismo video se ve que en la Figura \ref{fig:labeling13}, desaparece el objeto detectado, por otra parte el que utiliza el \textit{tracker} no pierde ninguna detección.

\vspace{5mm}
\begin{figure}[h]
	\centering
	\includegraphics[scale=0.3]{imagenes/deteccionvstracker2.png}
	\caption{Error al solo usar la detección en un proyecto.}
	\source{Elaboración propia.}
	\label{fig:labeling13}
\end{figure}
\vspace{5mm}

Pero el problema mas importante es que cuando una persona se sale del enfoque de la cámara o es cubierto por algún objeto, la detección se pierde y cuando vuelve a aparecer la app lo detecta como si fuera una nueva persona. 
Esto se podrá solucionar incorporando a la detección un \textit{tracker}, mediante esto se podrá generar ID a cada detección y con eso se podrá diferenciar entre detecciones, ya que todos tendrán un identificador que permitiría contar a las personas detectadas.
El \textit{tracker} a utilizar es DeepSort, el cual es de uso libre, para que funcione correctamente se tiene que usar una red pre-entrenada que ayuda al proceso del rastreo.
\clearpage

\subsection{Agregar contador y límite de detección}

Para que se pueda contar la cantidad de personas detectas se usara una función de Deepsort el cual genera una lista de las detecciones y mediante OpenCV se podrá mostrar en la pantalla del mismo video como se ve en la Figura \ref{fig:labeling14}.

\begin{figure}[h]
	\centering
	\includegraphics[scale=0.6]{imagenes/persona19.png}
	\caption{Contador de personas en un video.}
	\source{Elaboración propia}
	\label{fig:labeling14}
\end{figure}

El límite se encargará de reiniciar el contador y los cálculos de tiempo máximo de espera de un cliente en una fila. En la Figura \ref{fig:labeling35} se aprecia cómo se muestra el límite creado, esta línea es solo una representación visual, ya que las coordenadas de la línea son los importantes al momento de trabajar en la aplicación.


\begin{figure}[h]
	\centering
	\includegraphics[scale=0.4]{imagenes/persona20.png}
	\caption{Línea que limita las detecciones en un video.}
	\source{Elaboración propia}
	\label{fig:labeling35}
\end{figure}

\clearpage
Esto se logra gracias a OpenCV, ya que con él se dibujara una línea, el cual representa el límite para la detección y cuando un objeto cruce la línea ya no será tomando en cuenta. Esto funciona porque por cada detección genera unas coordenadas de la posición del objeto, teniendo esto se sabe cuándo cruza la línea.

\subsection{Alarma al superar el tiempo máximo}

Este apartado se encarga de informar cuando se sobrepase el tiempo máximo de espera definido, para saber esto se creó una lista donde se registra la hora en la que se detectó la persona, en la Figura \ref{fig:labeling16} se observa como está compuesta la lista, la cual guarda datos necesarios para realizar la alarma y el límite de detección.

\vspace{5mm}
\begin{figure}[h]
	\centering
	\includegraphics[scale=0.5]{imagenes/persona21.png}
	\caption{Lista que guarda datos cada vez que se realiza una nueva detección.}
	\source{Elaboración propia}
	\label{fig:labeling16}
\end{figure}
\vspace{5mm}

Teniendo esta lista podemos registrar la hora en la que fue detectado una persona y con ello al tener una lista donde se detectaron 3 personas nos permite calcular un tiempo promedio de espera y con ese dato se podrá comparar con el tiempo máximo de espera.
Si este tiempo promedio de espera es mayor que el tiempo máximo, se procederá a notificar mediante un \say{popup} que el cliente está esperando mucho y que se recomienda abrir una nueva caja.
\\

Otra manera en la que se puede activar la alarma es cuando por un determinado tiempo ni una persona llegó a cruzar el límite definido, de igual manera que en el anterior caso se activará un \say{popup} notificando un problema al encargado para determinar si lo ve necesario abrir una caja o no. En la Figura \ref{fig:labeling17} se observa el diseño del \say{popup} y el mensaje que da.

\vspace{5mm}
\begin{figure}[h]
	\centering
	\includegraphics[scale=0.45]{imagenes/persona22.png}
	\caption{Alerta popup que se activa cuando se alcanza el tiempo máximo de espera.}
	\source{Elaboración propia}
	\label{fig:labeling17}
\end{figure}


\subsection{Crear una base de datos}

Con este apartado se está utilizando una base datos para registrar la hora exacta cuando salte una alarma. Esto sirve para poder ver cuándo son las horas críticas y así buscar una solución que mejoraría la atención.
Para la base de datos se usa Mongodb, el cual te permite guardar los datos tanto de manera local como en la nube y es una herramienta gratuita.

En este caso como aplicación fue probada en un MacBook Pro 2014, la instalación de Mongodb variará en una máquina Windows. Simplemente para instalar la base de datos se tendrá que correr el siguiente comando que aparece en la Figura \ref{fig:labeling18}, estos comandos se tienen que hacer correr en la terminal.

\vspace{5mm}
\begin{figure}[h]
	\centering
	\includegraphics[scale=0.55]{imagenes/mongodc.png}
	\caption{Comando de instalación en Mac de MongoDB.}
	\source{Elaboración propia}
	\label{fig:labeling18}
\end{figure}

El dato que se guardará es la hora en la que la alarma se active, esto se logra con los siguientes comandos que aparecen en el siguiente Código \ref{24}, estos tienen que estar dentro de la aplicación creada.
\vspace{5mm}

 \begin{lstlisting}[language=Python,caption={Comando para guardar información en la base de datos.},captionpos=b,label=24]
post = {"data": datetime.now()}
alarms = db.alarms
alarms.insert_one(post)
DETECTION_EVENTS.clear()
\end{lstlisting}
\vspace{5mm}

En este caso la información de la base de datos se almacena de manera local, lo que significa que cada vez que se inicie la aplicación se tendrá que activar MongoDB, esto se logra con el siguiente comando que aparece en la Figura \ref{fig:labeling20}, de la misma manera se tendrá que desactivar cuando ya no se utilice la aplicación. 

\vspace{5mm}
\begin{figure}[h]
	\centering
	\includegraphics[scale=0.5]{imagenes/persona25.png}
	\caption{Comandos para iniciar y detener la base de datos.}
	\source{Elaboración propia}
	\label{fig:labeling20}
\end{figure}


\section{Conclusiones del capitulo III}

En el presente capítulo se explico el motivo de la elección de la red neuronal, la cual es YOLO. Teniendo echa la elección se decidio realizar pruebas para saber que version de YOLO se adapta mejor al proyecto y con esos resultados se opto por Tiny-Yolo.
\\
Con la red neuronal seleccionada se procedió a realizar el respectivo entrenamiento para que pueda detectar a personas y así poder utilizar la información obtenida para calcular el tiempo promedio de espera de las personas que están esperando en una fila. Posteriormente, se explico el funcionamiento de la aplicación y cómo el usuario puede interactuar con la interfaz.

   


















\chapter{Conclusiones}
\section{Conclusiones} 

\begin{itemize} 

	\item Se realizo la investigación de diferentes algoritmos para la detección de objetos en imágenes y videos, y se seleccionó el algoritmo de YOLOv4 porque su velocidad de detección es aceptable para los requerimientos de proyecto y no compromete demasiado la precisión al identificar a las personas. 

	\item Se entreno una red neuronal capaz de detectar personas en tiempo real utilizando imágenes y videos capturados por las cámaras de seguridad del supermercado. La red neuronal entrenada demostró tener una precisión del 71\% y un rendimiento del 63\% lo que es bastante aceptable tomando en cuenta que solo se utilizó 500 imágenes para el entrenamiento que se obtuvieron de las cámaras de seguridad. Esto se puede mejorar ampliando la cantidad y variedad de imágenes para el entrenamiento además que se identificó que el ángulo en la que se encuentra la cámara dificulta la detección de personas. 

	\item Se logro calcular el tiempo promedio de espera de las personas generando identificadores independientes para cada detección, esto permite poder contar los objetos que están siendo detectados y con ello poder calcular el tiempo promedio de espera. Para eliminar ciertas fallas que se tiene al trabajar solamente con la detección de objetos se agregó la función de rastreo Deepsort, el cual es uno de los más conocidos, ya que tiene uno de los mejores rendimientos al momento de realizar un rastreo de los objetos detectados. 

	\item Se implemento la aplicación de escritorio Neovision utilizando Python y Tkinter que obtiene las imágenes a través de OpenCv  para después ser analizadas por la red neuronal de YOLO que detecta a las personas y luego generar con un identificador único para cada persona. Con esta información la aplicación calcula el tiempo de espera promedio y presenta una alerta cuando se sobrepasa el tiempo máximo configurado.  

\end{itemize} 

\clearpage 

 

\section{Recomendaciones}  

Como recomendaciones técnicas, un aspecto muy importante es contar con el hardware adecuado para este tipo de sistemas, es decir, contar con una tarjeta de video (GPU) en la computadora donde se va a ejecutar el programa. El equipo que fue utilizado para este proyecto no cuenta con una GPU, con lo que conllevo a no tener el rendimiento adecuado, ya que todos los cálculos fueron ejecutados en el CPU. 

Otro apartado importante es tener en cuenta el tipo de cámara que se usará, ya que dependiendo de la calidad del video la exigencia de la aplicación será mayor, es decir, que si se trabaja con videos en 720p (1280x720 pixeles) consumirá menos recursos de la computadora que al trabajar con un video en 1080p (1920x1080 pixeles), porque la imagen es más pequeña y eso genera menos información al procesarlo.

Un apartado muy importante es definir bien la posición de la cámara de donde se sacará el video, ya que el lugar donde se coloque la cámara podría ayudar o perjudicar bastante al momento que la red neuronal realice las detecciones. 

Por otra parte, al momento de realizar el entrenamiento se recomienda usar una base de datos muy amplia y variada que le permita así obtener mejores resultados.

\section{Trabajos futuros} 

Uno de los trabajos a futuros es hacer uso de una base de datos que se almacene en la nube y no de manera local, esto facilitaría poder conocer los datos registrados, ya que se podría acceder desde cualquier computadora y así uno no se limita a usar la misma computadora para poder analizar los datos.   

Otro trabajo a futuro es la implementación de una aplicación móvil que se conecte con la base de datos que se encuentre en la nube, esta app notificaría a los trabajadores del supermercado cada vez que registre una alarma y con ello se podría solucionar más rápido el problema. 







\chapter{Recomendaciones}
Como recomendaciones técnicas, un aspecto muy importante es contar con el hardware adecuado para este tipo de sistemas, es decir, contar con una tarjeta de video (GPU) en la computadora donde se va ejecutar el programa. El equipo que fue utilizado para este proyecto no cuenta con una GPU, con lo que conllevo a no tener el rendimiento adecuado, ya que todos los cálculos fueron ejecutados en el CPU.

\vspace{5mm}
Otro apartado importante es tener en cuenta el tipo de cámara que se usara, ya que dependiendo de la calidad del video la exigencia de la aplicación será mayor, es decir, que si se trabaja con videos en 720p (1280x720 pixeles) consumirá menos recursos de la computadora que al trabajar con un video en 1080p (1920x1080 pixeles), porque la imagen es mas pequeño y eso genera menos información al procesarlo.

\bibliographystyle{unsrt}
\addcontentsline{toc}{chapter}{Bibliografia}
\bibliography{capitulos/bibliography.bib}

\end{document}